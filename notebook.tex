
% Default to the notebook output style

    


% Inherit from the specified cell style.




    
\documentclass[11pt]{article}

    
    
    \usepackage[T1]{fontenc}
    % Nicer default font (+ math font) than Computer Modern for most use cases
    \usepackage{mathpazo}

    % Basic figure setup, for now with no caption control since it's done
    % automatically by Pandoc (which extracts ![](path) syntax from Markdown).
    \usepackage{graphicx}
    % We will generate all images so they have a width \maxwidth. This means
    % that they will get their normal width if they fit onto the page, but
    % are scaled down if they would overflow the margins.
    \makeatletter
    \def\maxwidth{\ifdim\Gin@nat@width>\linewidth\linewidth
    \else\Gin@nat@width\fi}
    \makeatother
    \let\Oldincludegraphics\includegraphics
    % Set max figure width to be 80% of text width, for now hardcoded.
    \renewcommand{\includegraphics}[1]{\Oldincludegraphics[width=.8\maxwidth]{#1}}
    % Ensure that by default, figures have no caption (until we provide a
    % proper Figure object with a Caption API and a way to capture that
    % in the conversion process - todo).
    \usepackage{caption}
    \DeclareCaptionLabelFormat{nolabel}{}
    \captionsetup{labelformat=nolabel}

    \usepackage{adjustbox} % Used to constrain images to a maximum size 
    \usepackage{xcolor} % Allow colors to be defined
    \usepackage{enumerate} % Needed for markdown enumerations to work
    \usepackage{geometry} % Used to adjust the document margins
    \usepackage{amsmath} % Equations
    \usepackage{amssymb} % Equations
    \usepackage{textcomp} % defines textquotesingle
    % Hack from http://tex.stackexchange.com/a/47451/13684:
    \AtBeginDocument{%
        \def\PYZsq{\textquotesingle}% Upright quotes in Pygmentized code
    }
    \usepackage{upquote} % Upright quotes for verbatim code
    \usepackage{eurosym} % defines \euro
    \usepackage[mathletters]{ucs} % Extended unicode (utf-8) support
    \usepackage[utf8x]{inputenc} % Allow utf-8 characters in the tex document
    \usepackage{fancyvrb} % verbatim replacement that allows latex
    \usepackage{grffile} % extends the file name processing of package graphics 
                         % to support a larger range 
    % The hyperref package gives us a pdf with properly built
    % internal navigation ('pdf bookmarks' for the table of contents,
    % internal cross-reference links, web links for URLs, etc.)
    \usepackage{hyperref}
    \usepackage{longtable} % longtable support required by pandoc >1.10
    \usepackage{booktabs}  % table support for pandoc > 1.12.2
    \usepackage[inline]{enumitem} % IRkernel/repr support (it uses the enumerate* environment)
    \usepackage[normalem]{ulem} % ulem is needed to support strikethroughs (\sout)
                                % normalem makes italics be italics, not underlines
    

    
    
    % Colors for the hyperref package
    \definecolor{urlcolor}{rgb}{0,.145,.698}
    \definecolor{linkcolor}{rgb}{.71,0.21,0.01}
    \definecolor{citecolor}{rgb}{.12,.54,.11}

    % ANSI colors
    \definecolor{ansi-black}{HTML}{3E424D}
    \definecolor{ansi-black-intense}{HTML}{282C36}
    \definecolor{ansi-red}{HTML}{E75C58}
    \definecolor{ansi-red-intense}{HTML}{B22B31}
    \definecolor{ansi-green}{HTML}{00A250}
    \definecolor{ansi-green-intense}{HTML}{007427}
    \definecolor{ansi-yellow}{HTML}{DDB62B}
    \definecolor{ansi-yellow-intense}{HTML}{B27D12}
    \definecolor{ansi-blue}{HTML}{208FFB}
    \definecolor{ansi-blue-intense}{HTML}{0065CA}
    \definecolor{ansi-magenta}{HTML}{D160C4}
    \definecolor{ansi-magenta-intense}{HTML}{A03196}
    \definecolor{ansi-cyan}{HTML}{60C6C8}
    \definecolor{ansi-cyan-intense}{HTML}{258F8F}
    \definecolor{ansi-white}{HTML}{C5C1B4}
    \definecolor{ansi-white-intense}{HTML}{A1A6B2}

    % commands and environments needed by pandoc snippets
    % extracted from the output of `pandoc -s`
    \providecommand{\tightlist}{%
      \setlength{\itemsep}{0pt}\setlength{\parskip}{0pt}}
    \DefineVerbatimEnvironment{Highlighting}{Verbatim}{commandchars=\\\{\}}
    % Add ',fontsize=\small' for more characters per line
    \newenvironment{Shaded}{}{}
    \newcommand{\KeywordTok}[1]{\textcolor[rgb]{0.00,0.44,0.13}{\textbf{{#1}}}}
    \newcommand{\DataTypeTok}[1]{\textcolor[rgb]{0.56,0.13,0.00}{{#1}}}
    \newcommand{\DecValTok}[1]{\textcolor[rgb]{0.25,0.63,0.44}{{#1}}}
    \newcommand{\BaseNTok}[1]{\textcolor[rgb]{0.25,0.63,0.44}{{#1}}}
    \newcommand{\FloatTok}[1]{\textcolor[rgb]{0.25,0.63,0.44}{{#1}}}
    \newcommand{\CharTok}[1]{\textcolor[rgb]{0.25,0.44,0.63}{{#1}}}
    \newcommand{\StringTok}[1]{\textcolor[rgb]{0.25,0.44,0.63}{{#1}}}
    \newcommand{\CommentTok}[1]{\textcolor[rgb]{0.38,0.63,0.69}{\textit{{#1}}}}
    \newcommand{\OtherTok}[1]{\textcolor[rgb]{0.00,0.44,0.13}{{#1}}}
    \newcommand{\AlertTok}[1]{\textcolor[rgb]{1.00,0.00,0.00}{\textbf{{#1}}}}
    \newcommand{\FunctionTok}[1]{\textcolor[rgb]{0.02,0.16,0.49}{{#1}}}
    \newcommand{\RegionMarkerTok}[1]{{#1}}
    \newcommand{\ErrorTok}[1]{\textcolor[rgb]{1.00,0.00,0.00}{\textbf{{#1}}}}
    \newcommand{\NormalTok}[1]{{#1}}
    
    % Additional commands for more recent versions of Pandoc
    \newcommand{\ConstantTok}[1]{\textcolor[rgb]{0.53,0.00,0.00}{{#1}}}
    \newcommand{\SpecialCharTok}[1]{\textcolor[rgb]{0.25,0.44,0.63}{{#1}}}
    \newcommand{\VerbatimStringTok}[1]{\textcolor[rgb]{0.25,0.44,0.63}{{#1}}}
    \newcommand{\SpecialStringTok}[1]{\textcolor[rgb]{0.73,0.40,0.53}{{#1}}}
    \newcommand{\ImportTok}[1]{{#1}}
    \newcommand{\DocumentationTok}[1]{\textcolor[rgb]{0.73,0.13,0.13}{\textit{{#1}}}}
    \newcommand{\AnnotationTok}[1]{\textcolor[rgb]{0.38,0.63,0.69}{\textbf{\textit{{#1}}}}}
    \newcommand{\CommentVarTok}[1]{\textcolor[rgb]{0.38,0.63,0.69}{\textbf{\textit{{#1}}}}}
    \newcommand{\VariableTok}[1]{\textcolor[rgb]{0.10,0.09,0.49}{{#1}}}
    \newcommand{\ControlFlowTok}[1]{\textcolor[rgb]{0.00,0.44,0.13}{\textbf{{#1}}}}
    \newcommand{\OperatorTok}[1]{\textcolor[rgb]{0.40,0.40,0.40}{{#1}}}
    \newcommand{\BuiltInTok}[1]{{#1}}
    \newcommand{\ExtensionTok}[1]{{#1}}
    \newcommand{\PreprocessorTok}[1]{\textcolor[rgb]{0.74,0.48,0.00}{{#1}}}
    \newcommand{\AttributeTok}[1]{\textcolor[rgb]{0.49,0.56,0.16}{{#1}}}
    \newcommand{\InformationTok}[1]{\textcolor[rgb]{0.38,0.63,0.69}{\textbf{\textit{{#1}}}}}
    \newcommand{\WarningTok}[1]{\textcolor[rgb]{0.38,0.63,0.69}{\textbf{\textit{{#1}}}}}
    
    
    % Define a nice break command that doesn't care if a line doesn't already
    % exist.
    \def\br{\hspace*{\fill} \\* }
    % Math Jax compatability definitions
    \def\gt{>}
    \def\lt{<}
    % Document parameters
    \title{julia\_}
    
    
    

    % Pygments definitions
    
\makeatletter
\def\PY@reset{\let\PY@it=\relax \let\PY@bf=\relax%
    \let\PY@ul=\relax \let\PY@tc=\relax%
    \let\PY@bc=\relax \let\PY@ff=\relax}
\def\PY@tok#1{\csname PY@tok@#1\endcsname}
\def\PY@toks#1+{\ifx\relax#1\empty\else%
    \PY@tok{#1}\expandafter\PY@toks\fi}
\def\PY@do#1{\PY@bc{\PY@tc{\PY@ul{%
    \PY@it{\PY@bf{\PY@ff{#1}}}}}}}
\def\PY#1#2{\PY@reset\PY@toks#1+\relax+\PY@do{#2}}

\expandafter\def\csname PY@tok@w\endcsname{\def\PY@tc##1{\textcolor[rgb]{0.73,0.73,0.73}{##1}}}
\expandafter\def\csname PY@tok@c\endcsname{\let\PY@it=\textit\def\PY@tc##1{\textcolor[rgb]{0.25,0.50,0.50}{##1}}}
\expandafter\def\csname PY@tok@cp\endcsname{\def\PY@tc##1{\textcolor[rgb]{0.74,0.48,0.00}{##1}}}
\expandafter\def\csname PY@tok@k\endcsname{\let\PY@bf=\textbf\def\PY@tc##1{\textcolor[rgb]{0.00,0.50,0.00}{##1}}}
\expandafter\def\csname PY@tok@kp\endcsname{\def\PY@tc##1{\textcolor[rgb]{0.00,0.50,0.00}{##1}}}
\expandafter\def\csname PY@tok@kt\endcsname{\def\PY@tc##1{\textcolor[rgb]{0.69,0.00,0.25}{##1}}}
\expandafter\def\csname PY@tok@o\endcsname{\def\PY@tc##1{\textcolor[rgb]{0.40,0.40,0.40}{##1}}}
\expandafter\def\csname PY@tok@ow\endcsname{\let\PY@bf=\textbf\def\PY@tc##1{\textcolor[rgb]{0.67,0.13,1.00}{##1}}}
\expandafter\def\csname PY@tok@nb\endcsname{\def\PY@tc##1{\textcolor[rgb]{0.00,0.50,0.00}{##1}}}
\expandafter\def\csname PY@tok@nf\endcsname{\def\PY@tc##1{\textcolor[rgb]{0.00,0.00,1.00}{##1}}}
\expandafter\def\csname PY@tok@nc\endcsname{\let\PY@bf=\textbf\def\PY@tc##1{\textcolor[rgb]{0.00,0.00,1.00}{##1}}}
\expandafter\def\csname PY@tok@nn\endcsname{\let\PY@bf=\textbf\def\PY@tc##1{\textcolor[rgb]{0.00,0.00,1.00}{##1}}}
\expandafter\def\csname PY@tok@ne\endcsname{\let\PY@bf=\textbf\def\PY@tc##1{\textcolor[rgb]{0.82,0.25,0.23}{##1}}}
\expandafter\def\csname PY@tok@nv\endcsname{\def\PY@tc##1{\textcolor[rgb]{0.10,0.09,0.49}{##1}}}
\expandafter\def\csname PY@tok@no\endcsname{\def\PY@tc##1{\textcolor[rgb]{0.53,0.00,0.00}{##1}}}
\expandafter\def\csname PY@tok@nl\endcsname{\def\PY@tc##1{\textcolor[rgb]{0.63,0.63,0.00}{##1}}}
\expandafter\def\csname PY@tok@ni\endcsname{\let\PY@bf=\textbf\def\PY@tc##1{\textcolor[rgb]{0.60,0.60,0.60}{##1}}}
\expandafter\def\csname PY@tok@na\endcsname{\def\PY@tc##1{\textcolor[rgb]{0.49,0.56,0.16}{##1}}}
\expandafter\def\csname PY@tok@nt\endcsname{\let\PY@bf=\textbf\def\PY@tc##1{\textcolor[rgb]{0.00,0.50,0.00}{##1}}}
\expandafter\def\csname PY@tok@nd\endcsname{\def\PY@tc##1{\textcolor[rgb]{0.67,0.13,1.00}{##1}}}
\expandafter\def\csname PY@tok@s\endcsname{\def\PY@tc##1{\textcolor[rgb]{0.73,0.13,0.13}{##1}}}
\expandafter\def\csname PY@tok@sd\endcsname{\let\PY@it=\textit\def\PY@tc##1{\textcolor[rgb]{0.73,0.13,0.13}{##1}}}
\expandafter\def\csname PY@tok@si\endcsname{\let\PY@bf=\textbf\def\PY@tc##1{\textcolor[rgb]{0.73,0.40,0.53}{##1}}}
\expandafter\def\csname PY@tok@se\endcsname{\let\PY@bf=\textbf\def\PY@tc##1{\textcolor[rgb]{0.73,0.40,0.13}{##1}}}
\expandafter\def\csname PY@tok@sr\endcsname{\def\PY@tc##1{\textcolor[rgb]{0.73,0.40,0.53}{##1}}}
\expandafter\def\csname PY@tok@ss\endcsname{\def\PY@tc##1{\textcolor[rgb]{0.10,0.09,0.49}{##1}}}
\expandafter\def\csname PY@tok@sx\endcsname{\def\PY@tc##1{\textcolor[rgb]{0.00,0.50,0.00}{##1}}}
\expandafter\def\csname PY@tok@m\endcsname{\def\PY@tc##1{\textcolor[rgb]{0.40,0.40,0.40}{##1}}}
\expandafter\def\csname PY@tok@gh\endcsname{\let\PY@bf=\textbf\def\PY@tc##1{\textcolor[rgb]{0.00,0.00,0.50}{##1}}}
\expandafter\def\csname PY@tok@gu\endcsname{\let\PY@bf=\textbf\def\PY@tc##1{\textcolor[rgb]{0.50,0.00,0.50}{##1}}}
\expandafter\def\csname PY@tok@gd\endcsname{\def\PY@tc##1{\textcolor[rgb]{0.63,0.00,0.00}{##1}}}
\expandafter\def\csname PY@tok@gi\endcsname{\def\PY@tc##1{\textcolor[rgb]{0.00,0.63,0.00}{##1}}}
\expandafter\def\csname PY@tok@gr\endcsname{\def\PY@tc##1{\textcolor[rgb]{1.00,0.00,0.00}{##1}}}
\expandafter\def\csname PY@tok@ge\endcsname{\let\PY@it=\textit}
\expandafter\def\csname PY@tok@gs\endcsname{\let\PY@bf=\textbf}
\expandafter\def\csname PY@tok@gp\endcsname{\let\PY@bf=\textbf\def\PY@tc##1{\textcolor[rgb]{0.00,0.00,0.50}{##1}}}
\expandafter\def\csname PY@tok@go\endcsname{\def\PY@tc##1{\textcolor[rgb]{0.53,0.53,0.53}{##1}}}
\expandafter\def\csname PY@tok@gt\endcsname{\def\PY@tc##1{\textcolor[rgb]{0.00,0.27,0.87}{##1}}}
\expandafter\def\csname PY@tok@err\endcsname{\def\PY@bc##1{\setlength{\fboxsep}{0pt}\fcolorbox[rgb]{1.00,0.00,0.00}{1,1,1}{\strut ##1}}}
\expandafter\def\csname PY@tok@kc\endcsname{\let\PY@bf=\textbf\def\PY@tc##1{\textcolor[rgb]{0.00,0.50,0.00}{##1}}}
\expandafter\def\csname PY@tok@kd\endcsname{\let\PY@bf=\textbf\def\PY@tc##1{\textcolor[rgb]{0.00,0.50,0.00}{##1}}}
\expandafter\def\csname PY@tok@kn\endcsname{\let\PY@bf=\textbf\def\PY@tc##1{\textcolor[rgb]{0.00,0.50,0.00}{##1}}}
\expandafter\def\csname PY@tok@kr\endcsname{\let\PY@bf=\textbf\def\PY@tc##1{\textcolor[rgb]{0.00,0.50,0.00}{##1}}}
\expandafter\def\csname PY@tok@bp\endcsname{\def\PY@tc##1{\textcolor[rgb]{0.00,0.50,0.00}{##1}}}
\expandafter\def\csname PY@tok@fm\endcsname{\def\PY@tc##1{\textcolor[rgb]{0.00,0.00,1.00}{##1}}}
\expandafter\def\csname PY@tok@vc\endcsname{\def\PY@tc##1{\textcolor[rgb]{0.10,0.09,0.49}{##1}}}
\expandafter\def\csname PY@tok@vg\endcsname{\def\PY@tc##1{\textcolor[rgb]{0.10,0.09,0.49}{##1}}}
\expandafter\def\csname PY@tok@vi\endcsname{\def\PY@tc##1{\textcolor[rgb]{0.10,0.09,0.49}{##1}}}
\expandafter\def\csname PY@tok@vm\endcsname{\def\PY@tc##1{\textcolor[rgb]{0.10,0.09,0.49}{##1}}}
\expandafter\def\csname PY@tok@sa\endcsname{\def\PY@tc##1{\textcolor[rgb]{0.73,0.13,0.13}{##1}}}
\expandafter\def\csname PY@tok@sb\endcsname{\def\PY@tc##1{\textcolor[rgb]{0.73,0.13,0.13}{##1}}}
\expandafter\def\csname PY@tok@sc\endcsname{\def\PY@tc##1{\textcolor[rgb]{0.73,0.13,0.13}{##1}}}
\expandafter\def\csname PY@tok@dl\endcsname{\def\PY@tc##1{\textcolor[rgb]{0.73,0.13,0.13}{##1}}}
\expandafter\def\csname PY@tok@s2\endcsname{\def\PY@tc##1{\textcolor[rgb]{0.73,0.13,0.13}{##1}}}
\expandafter\def\csname PY@tok@sh\endcsname{\def\PY@tc##1{\textcolor[rgb]{0.73,0.13,0.13}{##1}}}
\expandafter\def\csname PY@tok@s1\endcsname{\def\PY@tc##1{\textcolor[rgb]{0.73,0.13,0.13}{##1}}}
\expandafter\def\csname PY@tok@mb\endcsname{\def\PY@tc##1{\textcolor[rgb]{0.40,0.40,0.40}{##1}}}
\expandafter\def\csname PY@tok@mf\endcsname{\def\PY@tc##1{\textcolor[rgb]{0.40,0.40,0.40}{##1}}}
\expandafter\def\csname PY@tok@mh\endcsname{\def\PY@tc##1{\textcolor[rgb]{0.40,0.40,0.40}{##1}}}
\expandafter\def\csname PY@tok@mi\endcsname{\def\PY@tc##1{\textcolor[rgb]{0.40,0.40,0.40}{##1}}}
\expandafter\def\csname PY@tok@il\endcsname{\def\PY@tc##1{\textcolor[rgb]{0.40,0.40,0.40}{##1}}}
\expandafter\def\csname PY@tok@mo\endcsname{\def\PY@tc##1{\textcolor[rgb]{0.40,0.40,0.40}{##1}}}
\expandafter\def\csname PY@tok@ch\endcsname{\let\PY@it=\textit\def\PY@tc##1{\textcolor[rgb]{0.25,0.50,0.50}{##1}}}
\expandafter\def\csname PY@tok@cm\endcsname{\let\PY@it=\textit\def\PY@tc##1{\textcolor[rgb]{0.25,0.50,0.50}{##1}}}
\expandafter\def\csname PY@tok@cpf\endcsname{\let\PY@it=\textit\def\PY@tc##1{\textcolor[rgb]{0.25,0.50,0.50}{##1}}}
\expandafter\def\csname PY@tok@c1\endcsname{\let\PY@it=\textit\def\PY@tc##1{\textcolor[rgb]{0.25,0.50,0.50}{##1}}}
\expandafter\def\csname PY@tok@cs\endcsname{\let\PY@it=\textit\def\PY@tc##1{\textcolor[rgb]{0.25,0.50,0.50}{##1}}}

\def\PYZbs{\char`\\}
\def\PYZus{\char`\_}
\def\PYZob{\char`\{}
\def\PYZcb{\char`\}}
\def\PYZca{\char`\^}
\def\PYZam{\char`\&}
\def\PYZlt{\char`\<}
\def\PYZgt{\char`\>}
\def\PYZsh{\char`\#}
\def\PYZpc{\char`\%}
\def\PYZdl{\char`\$}
\def\PYZhy{\char`\-}
\def\PYZsq{\char`\'}
\def\PYZdq{\char`\"}
\def\PYZti{\char`\~}
% for compatibility with earlier versions
\def\PYZat{@}
\def\PYZlb{[}
\def\PYZrb{]}
\makeatother


    % Exact colors from NB
    \definecolor{incolor}{rgb}{0.0, 0.0, 0.5}
    \definecolor{outcolor}{rgb}{0.545, 0.0, 0.0}



    
    % Prevent overflowing lines due to hard-to-break entities
    \sloppy 
    % Setup hyperref package
    \hypersetup{
      breaklinks=true,  % so long urls are correctly broken across lines
      colorlinks=true,
      urlcolor=urlcolor,
      linkcolor=linkcolor,
      citecolor=citecolor,
      }
    % Slightly bigger margins than the latex defaults
    
    \geometry{verbose,tmargin=1in,bmargin=1in,lmargin=1in,rmargin=1in}
    
    

    \begin{document}
    
    
    \maketitle
    
    

    
    \#

\begin{align}  Julia  \end{align}

\begin{center}\rule{0.5\linewidth}{\linethickness}\end{center}

    http://www.rpi.edu/dept/arc/docs/latex/latex-intro.pdf todo esto se
puede usar para darle formato al titulo

    

    The JuliaCon academic conference for Julia users and developers has been
held annually since 2014.

    Usualmente los lenguajes de programacion para computación numerica se
dividen entre:

\begin{itemize}
\item
  Estaticos como C que son rápidos en ejecucion pero lentos en el
  desarrollo.
\item
  Dinámicos como Python que son rápidos en el desarrollo pero lentos en
  ejecucuion.
\end{itemize}

Julia busca ofrecer lo mejor de los dos mundos.

Es un lenguaje nuevo, empezó a trabajarse en él en el 2009.

Sus creadores son Jeff Bezanson, Stefan Karpinski, Viral B. Shah y Alan
Edelman.

La primera versión pública se lanzó en el 2012.

En 2015 se establece Julia computing para proveer soporte y
entrenamiento pago.

Para Enero del 2018 tenía alrededor de 1.800.000 descargas.

    \section{Para que sirve}\label{para-que-sirve}

    Se trata de un lenguaje cuya concepción inicial tenía como objetivo
lograr desarrollar un lenguaje que soporte un gran manejo matemático no
sólo en capacidades computacionales sino también en facilidad y
versatilidad. Suele utilizarse para para la computación paralela y
distribuida gracias a su buen diseño y manejo de concurrencia. De la
misma manera, su capacidad matemática lo hace un lenguaje muy utilizado
en el ámbito de Machine Learning donde se trabaja con una enorme
cantidad de datos.

    \section{Caracteristicas basicas del
lenguaje}\label{caracteristicas-basicas-del-lenguaje}

    Tipos definidos por usuarios son tan rápidos y compctos como los
Built-ins

Llama funciones C directamente (no se necesitan wrappers ni API's
especiales)

Dispone de un compilador avanzado, JIT (Just In Time), que tiene como
objetivo la optimización del código en virtud de una mejora en el tiempo
de ejecución.

    \subsubsection{Variables}\label{variables}

    \begin{Verbatim}[commandchars=\\\{\}]
{\color{incolor}In [{\color{incolor} }]:} \PY{n}{miVariable} \PY{o}{=} \PY{l+m+mi}{1}
\end{Verbatim}


    \begin{Verbatim}[commandchars=\\\{\}]
{\color{incolor}In [{\color{incolor} }]:} \PY{n}{miVariable} \PY{o}{=} \PY{n}{miVariable} \PY{o}{+} \PY{l+m+mi}{5}
\end{Verbatim}


    \begin{Verbatim}[commandchars=\\\{\}]
{\color{incolor}In [{\color{incolor} }]:} \PY{n}{δ} \PY{o}{=} \PY{l+m+mf}{0.00001}
\end{Verbatim}


    \begin{Verbatim}[commandchars=\\\{\}]
{\color{incolor}In [{\color{incolor} }]:} \PY{n}{δ} \PY{o}{=} \PY{l+s}{\PYZdq{}}\PY{l+s}{H}\PY{l+s}{o}\PY{l+s}{l}\PY{l+s}{a}\PY{l+s}{ }\PY{l+s}{a}\PY{l+s}{ }\PY{l+s}{t}\PY{l+s}{o}\PY{l+s}{d}\PY{l+s}{o}\PY{l+s}{s}\PY{l+s}{ }\PY{l+s}{y}\PY{l+s}{ }\PY{l+s}{t}\PY{l+s}{o}\PY{l+s}{d}\PY{l+s}{a}\PY{l+s}{s}\PY{l+s}{\PYZdq{}}
\end{Verbatim}


    \begin{Verbatim}[commandchars=\\\{\}]
{\color{incolor}In [{\color{incolor} }]:} \PY{c}{\PYZsh{} Constantes matemáticas}
        \PY{n+nb}{π}
\end{Verbatim}


    \begin{Verbatim}[commandchars=\\\{\}]
{\color{incolor}In [{\color{incolor} }]:} \PY{n}{miVariable} \PY{o}{=} \PY{l+m+mi}{2}\PY{o}{*}\PY{n+nb}{π} \PY{o}{+} \PY{l+m+mf}{0.12}
\end{Verbatim}


    \begin{Verbatim}[commandchars=\\\{\}]
{\color{incolor}In [{\color{incolor} }]:} \PY{n}{miVariable} \PY{o}{=} \PY{n}{miVariable} \PY{o}{+} \PY{n+nb}{Inf}
\end{Verbatim}


    \begin{Verbatim}[commandchars=\\\{\}]
{\color{incolor}In [{\color{incolor} }]:} \PY{n}{miVariable} \PY{o}{=} \PY{n}{miVariable} \PY{o}{+} \PY{o}{\PYZhy{}}\PY{n+nb}{Inf}
\end{Verbatim}


    \begin{Verbatim}[commandchars=\\\{\}]
{\color{incolor}In [{\color{incolor} }]:} \PY{c}{\PYZsh{} soporte para números irracionales}
        \PY{n}{res} \PY{o}{=} \PY{p}{(}\PY{l+m+mi}{1} \PY{o}{+} \PY{l+m+mi}{4}\PY{n+nb}{im}\PY{p}{)}\PY{o}{*}\PY{p}{(}\PY{l+m+mi}{2} \PY{o}{\PYZhy{}} \PY{l+m+mi}{3}\PY{n+nb}{im}\PY{p}{)}
\end{Verbatim}


    \begin{Verbatim}[commandchars=\\\{\}]
{\color{incolor}In [{\color{incolor} }]:} \PY{n}{cos}\PY{p}{(}\PY{n}{res}\PY{p}{)}
\end{Verbatim}


    \begin{Verbatim}[commandchars=\\\{\}]
{\color{incolor}In [{\color{incolor} }]:} \PY{c}{\PYZsh{} A diferencia de otros  lenguajes que soportan el paradigma funcional, Julia permite redefinir no solamente }
        \PY{c}{\PYZsh{} variables sino también hasta sus propias constantes}
        \PY{n+nb}{π} \PY{o}{=} \PY{l+m+mf}{3.14}
\end{Verbatim}


    \begin{Verbatim}[commandchars=\\\{\}]
{\color{incolor}In [{\color{incolor}2}]:} \PY{n}{tup} \PY{o}{=} \PY{p}{(}\PY{l+m+mi}{1}\PY{p}{,} \PY{l+m+mi}{2}\PY{p}{,} \PY{l+m+mi}{3}\PY{p}{)} 
        \PY{n}{length}\PY{p}{(}\PY{n}{tup}\PY{p}{)}
\end{Verbatim}


\begin{Verbatim}[commandchars=\\\{\}]
{\color{outcolor}Out[{\color{outcolor}2}]:} 3
\end{Verbatim}
            
    \begin{Verbatim}[commandchars=\\\{\}]
{\color{incolor}In [{\color{incolor} }]:} \PY{k+kp}{in}\PY{p}{(}\PY{l+m+mi}{4}\PY{p}{,} \PY{n}{tup}\PY{p}{)} \PY{c}{\PYZsh{} =\PYZgt{} true}
\end{Verbatim}


    \begin{Verbatim}[commandchars=\\\{\}]
{\color{incolor}In [{\color{incolor} }]:} \PY{n}{d}\PY{p}{,} \PY{n+nb}{e}\PY{p}{,} \PY{n}{f} \PY{o}{=} \PY{l+m+mi}{4}\PY{p}{,} \PY{l+m+mi}{5}\PY{p}{,} \PY{l+m+mi}{6}
\end{Verbatim}


    \begin{Verbatim}[commandchars=\\\{\}]
{\color{incolor}In [{\color{incolor} }]:} \PY{n+nb}{e}\PY{p}{,} \PY{n}{d} \PY{o}{=} \PY{n}{d}\PY{p}{,} \PY{n+nb}{e}
        \PY{n+nb}{e}\PY{p}{,} \PY{n}{d}
\end{Verbatim}


    \begin{Verbatim}[commandchars=\\\{\}]
{\color{incolor}In [{\color{incolor} }]:} \PY{c}{\PYZsh{} números racionales}
        \PY{l+m+mi}{1}\PY{o}{//}\PY{l+m+mi}{3}
\end{Verbatim}


    \begin{Verbatim}[commandchars=\\\{\}]
{\color{incolor}In [{\color{incolor} }]:} \PY{n}{float}\PY{p}{(}\PY{l+m+mi}{1}\PY{o}{//}\PY{l+m+mi}{3}\PY{p}{)}
\end{Verbatim}


    \subsubsection{Tipado}\label{tipado}

En Julia los tipados son, por defecto, omitidos de manera tal que las
variables admitan valores de cualquier tipo. Y esto se da así: son los
valores los que tienen tipo mientras que las variables son simples
notaciones que hacen referencia a entidades. De todas formas, es fácil
expresar el tipo esperado del valor de una cierta variable. Esto suele
mejorar no solamente la robustez de los programas, sino también su
performance. En conclusión, Julia es un lenguaje de tipado dinámico pero
que soporta la especificación de tipos para, en tiempo de ejecución,
favorecer a cuestiones internas del lenguaje que mejoran su performance.

    \begin{Verbatim}[commandchars=\\\{\}]
{\color{incolor}In [{\color{incolor}15}]:} \PY{n}{x} \PY{o}{=} \PY{l+s}{\PYZdq{}}\PY{l+s}{S}\PY{l+s}{o}\PY{l+s}{y}\PY{l+s}{ }\PY{l+s}{u}\PY{l+s}{n}\PY{l+s}{ }\PY{l+s}{s}\PY{l+s}{t}\PY{l+s}{r}\PY{l+s}{i}\PY{l+s}{n}\PY{l+s}{g}\PY{l+s}{\PYZdq{}}
\end{Verbatim}


\begin{Verbatim}[commandchars=\\\{\}]
{\color{outcolor}Out[{\color{outcolor}15}]:} "Soy un string"
\end{Verbatim}
            
    \begin{Verbatim}[commandchars=\\\{\}]
{\color{incolor}In [{\color{incolor}16}]:} \PY{n}{y}\PY{o}{::}\PY{k+kt}{Int64}
         \PY{n}{y} \PY{o}{=} \PY{n}{x}
\end{Verbatim}


    \begin{Verbatim}[commandchars=\\\{\}]

        TypeError: typeassert: expected Int64, got ASCIIString

        

    \end{Verbatim}

    \subsubsection{Manejo de errores simple}\label{manejo-de-errores-simple}

    \begin{Verbatim}[commandchars=\\\{\}]
{\color{incolor}In [{\color{incolor}1}]:} \PY{n}{tup}\PY{p}{[}\PY{l+m+mi}{1}\PY{p}{]} \PY{c}{\PYZsh{} =\PYZgt{} 1}
        \PY{k}{try}\PY{o}{:}
            \PY{n}{tup}\PY{p}{[}\PY{l+m+mi}{1}\PY{p}{]} \PY{o}{=} \PY{l+m+mi}{3} \PY{c}{\PYZsh{} MethodError}
        \PY{k}{catch} \PY{n+nb}{e}
            \PY{n}{println}\PY{p}{(}\PY{n+nb}{e}\PY{p}{)}
        \PY{k}{end}
\end{Verbatim}


    \begin{Verbatim}[commandchars=\\\{\}]

        UndefVarError: tup not defined

        

    \end{Verbatim}

    \begin{Verbatim}[commandchars=\\\{\}]
{\color{incolor}In [{\color{incolor}1}]:} \PY{n}{ampliacion\PYZus{}sqrt}\PY{p}{(}\PY{n}{x}\PY{p}{)} \PY{o}{=}
        
        \PY{n}{x} \PY{o}{\PYZgt{}=} \PY{l+m+mi}{0} \PY{o}{?} \PY{n}{sqrt}\PY{p}{(}\PY{n}{x}\PY{p}{)} \PY{o}{:} \PY{n}{error}\PY{p}{(}\PY{l+s}{\PYZdq{}}\PY{l+s}{V}\PY{l+s}{a}\PY{l+s}{l}\PY{l+s}{o}\PY{l+s}{r}\PY{l+s}{ }\PY{l+s}{d}\PY{l+s}{e}\PY{l+s}{ }\PY{l+s}{x}\PY{l+s}{ }\PY{l+s}{n}\PY{l+s}{e}\PY{l+s}{g}\PY{l+s}{a}\PY{l+s}{t}\PY{l+s}{i}\PY{l+s}{v}\PY{l+s}{o}\PY{l+s}{ }\PY{l+s}{e}\PY{l+s}{s}\PY{l+s}{ }\PY{l+s}{i}\PY{l+s}{n}\PY{l+s}{v}\PY{l+s}{á}\PY{l+s}{l}\PY{l+s}{i}\PY{l+s}{d}\PY{l+s}{o}\PY{l+s}{\PYZdq{}}\PY{p}{)}
        
        \PY{n}{ampliacion\PYZus{}sqrt}\PY{p}{(}\PY{l+m+mi}{100}\PY{p}{)}
\end{Verbatim}


\begin{Verbatim}[commandchars=\\\{\}]
{\color{outcolor}Out[{\color{outcolor}1}]:} 10.0
\end{Verbatim}
            
    \begin{Verbatim}[commandchars=\\\{\}]
{\color{incolor}In [{\color{incolor}2}]:} \PY{n}{ampliacion\PYZus{}sqrt}\PY{p}{(}\PY{o}{\PYZhy{}}\PY{l+m+mi}{6}\PY{p}{)}
\end{Verbatim}


    \begin{Verbatim}[commandchars=\\\{\}]

        Valor de x negativo es inválido

        

         in ampliacion\_sqrt at ./In[1]:1

    \end{Verbatim}

    \subsubsection{Funciones y argumentos}\label{funciones-y-argumentos}

Los argumentos en Julia se pasan siguiendo una convención que suele
llamarse "pass-by-sharing". Esto significa que no se pasan por copia,
sino que los argumentos funcionan en sí mismos como una nueva variable
que tiene idéntico valor al que ha sido pasado. Así, la modificación de
variables mutables como Arrays será visible para la función caller (esto
mismo sucede en Lisps, Python, Ruby entre otros).

https://www.youtube.com/watch?v=1YFss\_4B\_o4

    \begin{Verbatim}[commandchars=\\\{\}]
{\color{incolor}In [{\color{incolor}3}]:} \PY{p}{(}\PY{n}{x} \PY{o}{\PYZhy{}}\PY{o}{\PYZgt{}} \PY{n}{x}\PY{o}{\PYZca{}}\PY{l+m+mi}{2} \PY{o}{+} \PY{l+m+mi}{2}\PY{n}{x} \PY{o}{\PYZhy{}} \PY{l+m+mi}{1}\PY{p}{)}\PY{p}{(}\PY{l+m+mi}{2}\PY{p}{)}
\end{Verbatim}


\begin{Verbatim}[commandchars=\\\{\}]
{\color{outcolor}Out[{\color{outcolor}3}]:} 7
\end{Verbatim}
            
    \begin{Verbatim}[commandchars=\\\{\}]
{\color{incolor}In [{\color{incolor} }]:} \PY{n}{map}\PY{p}{(}\PY{n}{x} \PY{o}{\PYZhy{}}\PY{o}{\PYZgt{}} \PY{n}{x}\PY{o}{\PYZca{}}\PY{l+m+mi}{2} \PY{o}{+} \PY{l+m+mi}{2}\PY{n}{x} \PY{o}{\PYZhy{}} \PY{l+m+mi}{1}\PY{p}{,} \PY{p}{[}\PY{l+m+mi}{1}\PY{p}{,}\PY{l+m+mi}{3}\PY{p}{,}\PY{o}{\PYZhy{}}\PY{l+m+mi}{1}\PY{p}{]}\PY{p}{)}
\end{Verbatim}


    \begin{Verbatim}[commandchars=\\\{\}]
{\color{incolor}In [{\color{incolor} }]:} \PY{k}{function} \PY{n}{foo}\PY{p}{(}\PY{n}{a}\PY{p}{,}\PY{n}{b}\PY{p}{)}
                   \PY{n}{a}\PY{o}{+}\PY{n}{b}\PY{p}{,} \PY{n}{a}\PY{o}{*}\PY{n}{b}
        \PY{k}{end}
        \PY{n}{x}\PY{p}{,} \PY{n}{y} \PY{o}{=} \PY{n}{foo}\PY{p}{(}\PY{l+m+mi}{2}\PY{p}{,}\PY{l+m+mi}{3}\PY{p}{)}
\end{Verbatim}


    \begin{Verbatim}[commandchars=\\\{\}]
{\color{incolor}In [{\color{incolor} }]:} \PY{k}{function} \PY{n}{default}\PY{p}{(}\PY{n}{a}\PY{p}{,}\PY{n}{b}\PY{p}{,}\PY{n}{x}\PY{o}{=}\PY{l+m+mi}{5}\PY{p}{,}\PY{n}{y}\PY{o}{=}\PY{l+m+mi}{6}\PY{p}{)}
            \PY{k}{return} \PY{l+s}{\PYZdq{}}\PY{l+s+si}{\PYZdl{}a}\PY{l+s}{ }\PY{l+s+si}{\PYZdl{}b}\PY{l+s}{ }\PY{l+s}{a}\PY{l+s}{n}\PY{l+s}{d}\PY{l+s}{ }\PY{l+s+si}{\PYZdl{}x}\PY{l+s}{ }\PY{l+s+si}{\PYZdl{}y}\PY{l+s}{\PYZdq{}}
        \PY{k}{end}
        
        \PY{n}{default}\PY{p}{(}\PY{l+s+sc}{\PYZsq{}h\PYZsq{}}\PY{p}{,}\PY{l+s+sc}{\PYZsq{}g\PYZsq{}}\PY{p}{)}
\end{Verbatim}


    \begin{Verbatim}[commandchars=\\\{\}]
{\color{incolor}In [{\color{incolor} }]:} \PY{n}{default}\PY{p}{(}\PY{l+s+sc}{\PYZsq{}h\PYZsq{}}\PY{p}{,} \PY{l+s+sc}{\PYZsq{}g\PYZsq{}}\PY{p}{,} \PY{l+s+sc}{\PYZsq{}i\PYZsq{}}\PY{p}{)}
\end{Verbatim}


    \subsubsection{Ducking type}\label{ducking-type}

Quiero mostrar que a julia no le importa el tipo de dato. Julia opera
con cualquier tipo de dato que tenga sentido

    \begin{Verbatim}[commandchars=\\\{\}]
{\color{incolor}In [{\color{incolor}1}]:} \PY{n}{f2}\PY{p}{(}\PY{n}{x}\PY{p}{)} \PY{o}{=} \PY{n}{x}\PY{o}{\PYZca{}}\PY{l+m+mi}{2}
\end{Verbatim}


\begin{Verbatim}[commandchars=\\\{\}]
{\color{outcolor}Out[{\color{outcolor}1}]:} f2 (generic function with 1 method)
\end{Verbatim}
            
    \begin{Verbatim}[commandchars=\\\{\}]
{\color{incolor}In [{\color{incolor}3}]:} \PY{n}{f2}\PY{p}{(}\PY{l+m+mi}{2}\PY{p}{)}
\end{Verbatim}


\begin{Verbatim}[commandchars=\\\{\}]
{\color{outcolor}Out[{\color{outcolor}3}]:} 4
\end{Verbatim}
            
    \begin{Verbatim}[commandchars=\\\{\}]
{\color{incolor}In [{\color{incolor}4}]:} \PY{n}{Ad} \PY{o}{=} \PY{n}{rand}\PY{p}{(}\PY{l+m+mi}{3}\PY{p}{,} \PY{l+m+mi}{3}\PY{p}{)}
        \PY{n}{f2}\PY{p}{(}\PY{n}{Ad}\PY{p}{)}
\end{Verbatim}


\begin{Verbatim}[commandchars=\\\{\}]
{\color{outcolor}Out[{\color{outcolor}4}]:} 3×3 Array\{Float64,2\}:
         1.23116  0.915372  0.746678
         1.50719  1.21557   0.948403
         1.52776  1.18283   1.00988 
\end{Verbatim}
            
    \begin{Verbatim}[commandchars=\\\{\}]
{\color{incolor}In [{\color{incolor}6}]:} \PY{n}{f2}\PY{p}{(}\PY{l+s}{\PYZdq{}}\PY{l+s}{h}\PY{l+s}{o}\PY{l+s}{l}\PY{l+s}{a}\PY{l+s}{\PYZdq{}}\PY{p}{)} \PY{c}{\PYZsh{}La multiplicacion de este parametro implica una concatenacion}
\end{Verbatim}


\begin{Verbatim}[commandchars=\\\{\}]
{\color{outcolor}Out[{\color{outcolor}6}]:} "holahola"
\end{Verbatim}
            
    \begin{Verbatim}[commandchars=\\\{\}]
{\color{incolor}In [{\color{incolor}7}]:} \PY{n}{v} \PY{o}{=} \PY{n}{rand}\PY{p}{(}\PY{l+m+mi}{3}\PY{p}{)}
        \PY{n}{f2}\PY{p}{(}\PY{n}{v}\PY{p}{)} \PY{c}{\PYZsh{}Aca no funcionaria porque la operacion definida en f2 es ambigua. Hay diferentes manera de multiplicar un vector}
\end{Verbatim}


    \begin{Verbatim}[commandchars=\\\{\}]

        DimensionMismatch("Cannot multiply two vectors")

        

        Stacktrace:

         [1] power\_by\_squaring(::Array\{Float64,1\}, ::Int64) at ./intfuncs.jl:169

         [2] f2(::Array\{Float64,1\}) at ./In[1]:1

    \end{Verbatim}

    \subsubsection{Funciones mutantes vs no
mutantes}\label{funciones-mutantes-vs-no-mutantes}

Por convencion las funciones seguidas por un ! alteran, o bien mutan,
sus contenidos y las que carecen de un ! no lo hacen

    \begin{Verbatim}[commandchars=\\\{\}]
{\color{incolor}In [{\color{incolor}8}]:} \PY{n}{v} \PY{o}{=} \PY{p}{[}\PY{l+m+mi}{4}\PY{p}{,} \PY{l+m+mi}{7}\PY{p}{,} \PY{l+m+mi}{1}\PY{p}{]}
\end{Verbatim}


\begin{Verbatim}[commandchars=\\\{\}]
{\color{outcolor}Out[{\color{outcolor}8}]:} 3-element Array\{Int64,1\}:
         4
         7
         1
\end{Verbatim}
            
    \begin{Verbatim}[commandchars=\\\{\}]
{\color{incolor}In [{\color{incolor}9}]:} \PY{n}{sort}\PY{p}{(}\PY{n}{v}\PY{p}{)}
\end{Verbatim}


\begin{Verbatim}[commandchars=\\\{\}]
{\color{outcolor}Out[{\color{outcolor}9}]:} 3-element Array\{Int64,1\}:
         1
         4
         7
\end{Verbatim}
            
    \begin{Verbatim}[commandchars=\\\{\}]
{\color{incolor}In [{\color{incolor}10}]:} \PY{n}{v}
\end{Verbatim}


\begin{Verbatim}[commandchars=\\\{\}]
{\color{outcolor}Out[{\color{outcolor}10}]:} 3-element Array\{Int64,1\}:
          4
          7
          1
\end{Verbatim}
            
    \begin{Verbatim}[commandchars=\\\{\}]
{\color{incolor}In [{\color{incolor}11}]:} \PY{n}{sort!}\PY{p}{(}\PY{n}{v}\PY{p}{)}
\end{Verbatim}


\begin{Verbatim}[commandchars=\\\{\}]
{\color{outcolor}Out[{\color{outcolor}11}]:} 3-element Array\{Int64,1\}:
          1
          4
          7
\end{Verbatim}
            
    \begin{Verbatim}[commandchars=\\\{\}]
{\color{incolor}In [{\color{incolor}12}]:} \PY{n}{v}
\end{Verbatim}


\begin{Verbatim}[commandchars=\\\{\}]
{\color{outcolor}Out[{\color{outcolor}12}]:} 3-element Array\{Int64,1\}:
          1
          4
          7
\end{Verbatim}
            
    \subsubsection{Broadcasting}\label{broadcasting}

Si ponemos un . entre el nombre de la funcion y su lista de argumento le
estamos diciendo a la funcion que se aplique sobre cada elemento sobre
cada elemento del input. Esto es nativo de Julia y sirve para cualquier
funcion.

    \begin{Verbatim}[commandchars=\\\{\}]
{\color{incolor}In [{\color{incolor}13}]:} \PY{n}{Am} \PY{o}{=} \PY{p}{[}\PY{n}{i} \PY{o}{+} \PY{l+m+mi}{3}\PY{o}{*}\PY{n}{j} \PY{k}{for} \PY{n}{j} \PY{k+kp}{in} \PY{l+m+mi}{0}\PY{o}{:}\PY{l+m+mi}{2}\PY{p}{,} \PY{n}{i} \PY{k+kp}{in} \PY{l+m+mi}{1}\PY{o}{:}\PY{l+m+mi}{3}\PY{p}{]}
\end{Verbatim}


\begin{Verbatim}[commandchars=\\\{\}]
{\color{outcolor}Out[{\color{outcolor}13}]:} 3×3 Array\{Int64,2\}:
          1  2  3
          4  5  6
          7  8  9
\end{Verbatim}
            
    \begin{Verbatim}[commandchars=\\\{\}]
{\color{incolor}In [{\color{incolor}14}]:} \PY{n}{f2}\PY{p}{(}\PY{n}{Am}\PY{p}{)}  \PY{c}{\PYZsh{}Esto seria A\PYZca{}2 = A*A}
\end{Verbatim}


\begin{Verbatim}[commandchars=\\\{\}]
{\color{outcolor}Out[{\color{outcolor}14}]:} 3×3 Array\{Int64,2\}:
           30   36   42
           66   81   96
          102  126  150
\end{Verbatim}
            
    \begin{Verbatim}[commandchars=\\\{\}]
{\color{incolor}In [{\color{incolor}16}]:} \PY{n}{f2}\PY{o}{.}\PY{p}{(}\PY{n}{Am}\PY{p}{)} \PY{c}{\PYZsh{}Esto en cambio aplica x\PYZca{}2 a cada elemento de la matriz}
\end{Verbatim}


\begin{Verbatim}[commandchars=\\\{\}]
{\color{outcolor}Out[{\color{outcolor}16}]:} 3×3 Array\{Int64,2\}:
           1   4   9
          16  25  36
          49  64  81
\end{Verbatim}
            
    \subsubsection{Multiple Dispatch}\label{multiple-dispatch}

rapido, extensible, proglamable facilmente

    \begin{Verbatim}[commandchars=\\\{\}]
{\color{incolor}In [{\color{incolor}18}]:} \PY{c}{\PYZsh{}Para entender el despacho multiple en Julia, observemos el operador +}
         \PY{c}{\PYZsh{}Si llamamos a la funcion methods() sobre +, podemos ver todas las definiciones de +}
         \PY{n}{methods}\PY{p}{(}\PY{o}{+}\PY{p}{)}
\end{Verbatim}


\begin{Verbatim}[commandchars=\\\{\}]
{\color{outcolor}Out[{\color{outcolor}18}]:} \# 180 methods for generic function "+":
         +(x::Bool, z::Complex\{Bool\}) in Base at complex.jl:232
         +(x::Bool, y::Bool) in Base at bool.jl:89
         +(x::Bool) in Base at bool.jl:86
         +(x::Bool, y::T) where T<:AbstractFloat in Base at bool.jl:96
         +(x::Bool, z::Complex) in Base at complex.jl:239
         +(a::Float16, b::Float16) in Base at float.jl:372
         +(x::Float32, y::Float32) in Base at float.jl:374
         +(x::Float64, y::Float64) in Base at float.jl:375
         +(z::Complex\{Bool\}, x::Bool) in Base at complex.jl:233
         +(z::Complex\{Bool\}, x::Real) in Base at complex.jl:247
         +(x::Char, y::Integer) in Base at char.jl:40
         +(c::BigInt, x::BigFloat) in Base.MPFR at mpfr.jl:312
         +(a::BigInt, b::BigInt, c::BigInt, d::BigInt, e::BigInt) in Base.GMP at gmp.jl:334
         +(a::BigInt, b::BigInt, c::BigInt, d::BigInt) in Base.GMP at gmp.jl:327
         +(a::BigInt, b::BigInt, c::BigInt) in Base.GMP at gmp.jl:321
         +(x::BigInt, y::BigInt) in Base.GMP at gmp.jl:289
         +(x::BigInt, c::Union\{UInt16, UInt32, UInt64, UInt8\}) in Base.GMP at gmp.jl:346
         +(x::BigInt, c::Union\{Int16, Int32, Int64, Int8\}) in Base.GMP at gmp.jl:362
         +(a::BigFloat, b::BigFloat, c::BigFloat, d::BigFloat, e::BigFloat) in Base.MPFR at mpfr.jl:460
         +(a::BigFloat, b::BigFloat, c::BigFloat, d::BigFloat) in Base.MPFR at mpfr.jl:453
         +(a::BigFloat, b::BigFloat, c::BigFloat) in Base.MPFR at mpfr.jl:447
         +(x::BigFloat, c::BigInt) in Base.MPFR at mpfr.jl:308
         +(x::BigFloat, y::BigFloat) in Base.MPFR at mpfr.jl:277
         +(x::BigFloat, c::Union\{UInt16, UInt32, UInt64, UInt8\}) in Base.MPFR at mpfr.jl:284
         +(x::BigFloat, c::Union\{Int16, Int32, Int64, Int8\}) in Base.MPFR at mpfr.jl:292
         +(x::BigFloat, c::Union\{Float16, Float32, Float64\}) in Base.MPFR at mpfr.jl:300
         +(B::BitArray\{2\}, J::UniformScaling) in Base.LinAlg at linalg/uniformscaling.jl:59
         +(a::Base.Pkg.Resolve.VersionWeights.VWPreBuildItem, b::Base.Pkg.Resolve.VersionWeights.VWPreBuildItem) in Base.Pkg.Resolve.VersionWeights at pkg/resolve/versionweight.jl:87
         +(a::Base.Pkg.Resolve.VersionWeights.VWPreBuild, b::Base.Pkg.Resolve.VersionWeights.VWPreBuild) in Base.Pkg.Resolve.VersionWeights at pkg/resolve/versionweight.jl:129
         +(a::Base.Pkg.Resolve.VersionWeights.VersionWeight, b::Base.Pkg.Resolve.VersionWeights.VersionWeight) in Base.Pkg.Resolve.VersionWeights at pkg/resolve/versionweight.jl:191
         +(a::Base.Pkg.Resolve.MaxSum.FieldValues.FieldValue, b::Base.Pkg.Resolve.MaxSum.FieldValues.FieldValue) in Base.Pkg.Resolve.MaxSum.FieldValues at pkg/resolve/fieldvalue.jl:60
         +(x::Base.Dates.CompoundPeriod, y::Base.Dates.CompoundPeriod) in Base.Dates at dates/periods.jl:349
         +(x::Base.Dates.CompoundPeriod, y::Base.Dates.Period) in Base.Dates at dates/periods.jl:347
         +(x::Base.Dates.CompoundPeriod, y::Base.Dates.TimeType) in Base.Dates at dates/periods.jl:387
         +(x::Date, y::Base.Dates.Day) in Base.Dates at dates/arithmetic.jl:77
         +(x::Date, y::Base.Dates.Week) in Base.Dates at dates/arithmetic.jl:75
         +(dt::Date, z::Base.Dates.Month) in Base.Dates at dates/arithmetic.jl:58
         +(dt::Date, y::Base.Dates.Year) in Base.Dates at dates/arithmetic.jl:32
         +(dt::Date, t::Base.Dates.Time) in Base.Dates at dates/arithmetic.jl:20
         +(t::Base.Dates.Time, dt::Date) in Base.Dates at dates/arithmetic.jl:24
         +(x::Base.Dates.Time, y::Base.Dates.TimePeriod) in Base.Dates at dates/arithmetic.jl:81
         +(dt::DateTime, z::Base.Dates.Month) in Base.Dates at dates/arithmetic.jl:52
         +(dt::DateTime, y::Base.Dates.Year) in Base.Dates at dates/arithmetic.jl:28
         +(x::DateTime, y::Base.Dates.Period) in Base.Dates at dates/arithmetic.jl:79
         +(y::AbstractFloat, x::Bool) in Base at bool.jl:98
         +(x::T, y::T) where T<:Union\{Int128, Int16, Int32, Int64, Int8, UInt128, UInt16, UInt32, UInt64, UInt8\} in Base at int.jl:32
         +(x::Integer, y::Ptr) in Base at pointer.jl:128
         +(z::Complex, w::Complex) in Base at complex.jl:221
         +(z::Complex, x::Bool) in Base at complex.jl:240
         +(x::Real, z::Complex\{Bool\}) in Base at complex.jl:246
         +(x::Real, z::Complex) in Base at complex.jl:258
         +(z::Complex, x::Real) in Base at complex.jl:259
         +(x::Rational, y::Rational) in Base at rational.jl:245
         +(x::Integer, y::Char) in Base at char.jl:41
         +(i::Integer, index::CartesianIndex) in Base.IteratorsMD at multidimensional.jl:110
         +(c::Union\{UInt16, UInt32, UInt64, UInt8\}, x::BigInt) in Base.GMP at gmp.jl:350
         +(c::Union\{Int16, Int32, Int64, Int8\}, x::BigInt) in Base.GMP at gmp.jl:363
         +(c::Union\{UInt16, UInt32, UInt64, UInt8\}, x::BigFloat) in Base.MPFR at mpfr.jl:288
         +(c::Union\{Int16, Int32, Int64, Int8\}, x::BigFloat) in Base.MPFR at mpfr.jl:296
         +(c::Union\{Float16, Float32, Float64\}, x::BigFloat) in Base.MPFR at mpfr.jl:304
         +(x::Irrational, y::Irrational) in Base at irrationals.jl:109
         +(x::Real, r::Base.Use\_StepRangeLen\_Instead) in Base at deprecated.jl:1232
         +(x::Number) in Base at operators.jl:399
         +(x::T, y::T) where T<:Number in Base at promotion.jl:335
         +(x::Number, y::Number) in Base at promotion.jl:249
         +(x::Real, r::AbstractUnitRange) in Base at range.jl:721
         +(x::Number, r::AbstractUnitRange) in Base at range.jl:723
         +(x::Number, r::StepRangeLen) in Base at range.jl:726
         +(x::Number, r::LinSpace) in Base at range.jl:730
         +(x::Number, r::Range) in Base at range.jl:724
         +(r::Range, x::Number) in Base at range.jl:732
         +(r1::OrdinalRange, r2::OrdinalRange) in Base at range.jl:882
         +(r1::LinSpace\{T\}, r2::LinSpace\{T\}) where T in Base at range.jl:889
         +(r1::StepRangeLen\{T,R,S\} where S, r2::StepRangeLen\{T,R,S\} where S) where \{R<:Base.TwicePrecision, T\} in Base at twiceprecision.jl:300
         +(r1::StepRangeLen\{T,S,S\} where S, r2::StepRangeLen\{T,S,S\} where S) where \{T, S\} in Base at range.jl:905
         +(r1::Union\{LinSpace, OrdinalRange, StepRangeLen\}, r2::Union\{LinSpace, OrdinalRange, StepRangeLen\}) in Base at range.jl:897
         +(x::Base.TwicePrecision, y::Number) in Base at twiceprecision.jl:455
         +(x::Number, y::Base.TwicePrecision) in Base at twiceprecision.jl:458
         +(x::Base.TwicePrecision\{T\}, y::Base.TwicePrecision\{T\}) where T in Base at twiceprecision.jl:461
         +(x::Base.TwicePrecision, y::Base.TwicePrecision) in Base at twiceprecision.jl:465
         +(x::Ptr, y::Integer) in Base at pointer.jl:126
         +(A::BitArray, B::BitArray) in Base at bitarray.jl:1177
         +(A::SymTridiagonal, B::SymTridiagonal) in Base.LinAlg at linalg/tridiag.jl:128
         +(A::Tridiagonal, B::Tridiagonal) in Base.LinAlg at linalg/tridiag.jl:629
         +(A::UpperTriangular, B::UpperTriangular) in Base.LinAlg at linalg/triangular.jl:419
         +(A::LowerTriangular, B::LowerTriangular) in Base.LinAlg at linalg/triangular.jl:420
         +(A::UpperTriangular, B::Base.LinAlg.UnitUpperTriangular) in Base.LinAlg at linalg/triangular.jl:421
         +(A::LowerTriangular, B::Base.LinAlg.UnitLowerTriangular) in Base.LinAlg at linalg/triangular.jl:422
         +(A::Base.LinAlg.UnitUpperTriangular, B::UpperTriangular) in Base.LinAlg at linalg/triangular.jl:423
         +(A::Base.LinAlg.UnitLowerTriangular, B::LowerTriangular) in Base.LinAlg at linalg/triangular.jl:424
         +(A::Base.LinAlg.UnitUpperTriangular, B::Base.LinAlg.UnitUpperTriangular) in Base.LinAlg at linalg/triangular.jl:425
         +(A::Base.LinAlg.UnitLowerTriangular, B::Base.LinAlg.UnitLowerTriangular) in Base.LinAlg at linalg/triangular.jl:426
         +(A::Base.LinAlg.AbstractTriangular, B::Base.LinAlg.AbstractTriangular) in Base.LinAlg at linalg/triangular.jl:427
         +(A::Symmetric, x::Bool) in Base.LinAlg at linalg/symmetric.jl:274
         +(A::Symmetric, x::Number) in Base.LinAlg at linalg/symmetric.jl:276
         +(A::Hermitian, x::Bool) in Base.LinAlg at linalg/symmetric.jl:274
         +(A::Hermitian, x::Real) in Base.LinAlg at linalg/symmetric.jl:276
         +(Da::Diagonal, Db::Diagonal) in Base.LinAlg at linalg/diagonal.jl:140
         +(A::Bidiagonal, B::Bidiagonal) in Base.LinAlg at linalg/bidiag.jl:330
         +(UL::UpperTriangular, J::UniformScaling) in Base.LinAlg at linalg/uniformscaling.jl:72
         +(UL::Base.LinAlg.UnitUpperTriangular, J::UniformScaling) in Base.LinAlg at linalg/uniformscaling.jl:75
         +(UL::LowerTriangular, J::UniformScaling) in Base.LinAlg at linalg/uniformscaling.jl:72
         +(UL::Base.LinAlg.UnitLowerTriangular, J::UniformScaling) in Base.LinAlg at linalg/uniformscaling.jl:75
         +(A::Array, B::SparseMatrixCSC) in Base.SparseArrays at sparse/sparsematrix.jl:1541
         +(x::Union\{Base.ReshapedArray\{T,1,A,MI\} where MI<:Tuple\{Vararg\{Base.MultiplicativeInverses.SignedMultiplicativeInverse\{Int64\},N\} where N\} where A<:Union\{DenseArray, SubArray\{T,N,P,I,true\} where I<:Tuple\{Union\{Base.Slice, UnitRange\},Vararg\{Any,N\} where N\} where P where N where T\}, DenseArray\{T,1\}, SubArray\{T,1,A,I,L\} where L\} where I<:Tuple\{Vararg\{Union\{Base.AbstractCartesianIndex, Int64, Range\{Int64\}\},N\} where N\} where A<:Union\{Base.ReshapedArray\{T,N,A,MI\} where MI<:Tuple\{Vararg\{Base.MultiplicativeInverses.SignedMultiplicativeInverse\{Int64\},N\} where N\} where A<:Union\{DenseArray, SubArray\{T,N,P,I,true\} where I<:Tuple\{Union\{Base.Slice, UnitRange\},Vararg\{Any,N\} where N\} where P where N where T\} where N where T, DenseArray\} where T, y::AbstractSparseArray\{Tv,Ti,1\} where Ti where Tv) in Base.SparseArrays at sparse/sparsevector.jl:1340
         +(x::Union\{Base.ReshapedArray\{\#s268,N,A,MI\} where MI<:Tuple\{Vararg\{Base.MultiplicativeInverses.SignedMultiplicativeInverse\{Int64\},N\} where N\} where A<:Union\{DenseArray, SubArray\{T,N,P,I,true\} where I<:Tuple\{Union\{Base.Slice, UnitRange\},Vararg\{Any,N\} where N\} where P where N where T\}, DenseArray\{\#s268,N\}, SubArray\{\#s268,N,A,I,L\} where L\} where I<:Tuple\{Vararg\{Union\{Base.AbstractCartesianIndex, Int64, Range\{Int64\}\},N\} where N\} where A<:Union\{Base.ReshapedArray\{T,N,A,MI\} where MI<:Tuple\{Vararg\{Base.MultiplicativeInverses.SignedMultiplicativeInverse\{Int64\},N\} where N\} where A<:Union\{DenseArray, SubArray\{T,N,P,I,true\} where I<:Tuple\{Union\{Base.Slice, UnitRange\},Vararg\{Any,N\} where N\} where P where N where T\} where N where T, DenseArray\} where N where \#s268<:Union\{Base.Dates.CompoundPeriod, Base.Dates.Period\}) in Base.Dates at dates/periods.jl:358
         +(A::SparseMatrixCSC, J::UniformScaling) in Base.SparseArrays at sparse/sparsematrix.jl:3608
         +(A::AbstractArray\{TA,2\}, J::UniformScaling\{TJ\}) where \{TA, TJ\} in Base.LinAlg at linalg/uniformscaling.jl:119
         +(A::Diagonal, B::Bidiagonal) in Base.LinAlg at linalg/special.jl:113
         +(A::Bidiagonal, B::Diagonal) in Base.LinAlg at linalg/special.jl:114
         +(A::Diagonal, B::Tridiagonal) in Base.LinAlg at linalg/special.jl:113
         +(A::Tridiagonal, B::Diagonal) in Base.LinAlg at linalg/special.jl:114
         +(A::Diagonal, B::Array\{T,2\} where T) in Base.LinAlg at linalg/special.jl:113
         +(A::Array\{T,2\} where T, B::Diagonal) in Base.LinAlg at linalg/special.jl:114
         +(A::Bidiagonal, B::Tridiagonal) in Base.LinAlg at linalg/special.jl:113
         +(A::Tridiagonal, B::Bidiagonal) in Base.LinAlg at linalg/special.jl:114
         +(A::Bidiagonal, B::Array\{T,2\} where T) in Base.LinAlg at linalg/special.jl:113
         +(A::Array\{T,2\} where T, B::Bidiagonal) in Base.LinAlg at linalg/special.jl:114
         +(A::Tridiagonal, B::Array\{T,2\} where T) in Base.LinAlg at linalg/special.jl:113
         +(A::Array\{T,2\} where T, B::Tridiagonal) in Base.LinAlg at linalg/special.jl:114
         +(A::SymTridiagonal, B::Tridiagonal) in Base.LinAlg at linalg/special.jl:122
         +(A::Tridiagonal, B::SymTridiagonal) in Base.LinAlg at linalg/special.jl:123
         +(A::SymTridiagonal, B::Array\{T,2\} where T) in Base.LinAlg at linalg/special.jl:122
         +(A::Array\{T,2\} where T, B::SymTridiagonal) in Base.LinAlg at linalg/special.jl:123
         +(A::Diagonal, B::SymTridiagonal) in Base.LinAlg at linalg/special.jl:131
         +(A::SymTridiagonal, B::Diagonal) in Base.LinAlg at linalg/special.jl:132
         +(A::Bidiagonal, B::SymTridiagonal) in Base.LinAlg at linalg/special.jl:131
         +(A::SymTridiagonal, B::Bidiagonal) in Base.LinAlg at linalg/special.jl:132
         +(A::Diagonal, B::UpperTriangular) in Base.LinAlg at linalg/special.jl:143
         +(A::UpperTriangular, B::Diagonal) in Base.LinAlg at linalg/special.jl:144
         +(A::Diagonal, B::Base.LinAlg.UnitUpperTriangular) in Base.LinAlg at linalg/special.jl:143
         +(A::Base.LinAlg.UnitUpperTriangular, B::Diagonal) in Base.LinAlg at linalg/special.jl:144
         +(A::Diagonal, B::LowerTriangular) in Base.LinAlg at linalg/special.jl:143
         +(A::LowerTriangular, B::Diagonal) in Base.LinAlg at linalg/special.jl:144
         +(A::Diagonal, B::Base.LinAlg.UnitLowerTriangular) in Base.LinAlg at linalg/special.jl:143
         +(A::Base.LinAlg.UnitLowerTriangular, B::Diagonal) in Base.LinAlg at linalg/special.jl:144
         +(A::Base.LinAlg.AbstractTriangular, B::SymTridiagonal) in Base.LinAlg at linalg/special.jl:150
         +(A::SymTridiagonal, B::Base.LinAlg.AbstractTriangular) in Base.LinAlg at linalg/special.jl:151
         +(A::Base.LinAlg.AbstractTriangular, B::Tridiagonal) in Base.LinAlg at linalg/special.jl:150
         +(A::Tridiagonal, B::Base.LinAlg.AbstractTriangular) in Base.LinAlg at linalg/special.jl:151
         +(A::Base.LinAlg.AbstractTriangular, B::Bidiagonal) in Base.LinAlg at linalg/special.jl:150
         +(A::Bidiagonal, B::Base.LinAlg.AbstractTriangular) in Base.LinAlg at linalg/special.jl:151
         +(A::Base.LinAlg.AbstractTriangular, B::Array\{T,2\} where T) in Base.LinAlg at linalg/special.jl:150
         +(A::Array\{T,2\} where T, B::Base.LinAlg.AbstractTriangular) in Base.LinAlg at linalg/special.jl:151
         +(Y::Union\{Base.ReshapedArray\{\#s267,N,A,MI\} where MI<:Tuple\{Vararg\{Base.MultiplicativeInverses.SignedMultiplicativeInverse\{Int64\},N\} where N\} where A<:Union\{DenseArray, SubArray\{T,N,P,I,true\} where I<:Tuple\{Union\{Base.Slice, UnitRange\},Vararg\{Any,N\} where N\} where P where N where T\}, DenseArray\{\#s267,N\}, SubArray\{\#s267,N,A,I,L\} where L\} where I<:Tuple\{Vararg\{Union\{Base.AbstractCartesianIndex, Int64, Range\{Int64\}\},N\} where N\} where A<:Union\{Base.ReshapedArray\{T,N,A,MI\} where MI<:Tuple\{Vararg\{Base.MultiplicativeInverses.SignedMultiplicativeInverse\{Int64\},N\} where N\} where A<:Union\{DenseArray, SubArray\{T,N,P,I,true\} where I<:Tuple\{Union\{Base.Slice, UnitRange\},Vararg\{Any,N\} where N\} where P where N where T\} where N where T, DenseArray\} where N where \#s267<:Union\{Base.Dates.CompoundPeriod, Base.Dates.Period\}, x::Union\{Base.Dates.CompoundPeriod, Base.Dates.Period\}) in Base.Dates at dates/periods.jl:363
         +(X::Union\{Base.ReshapedArray\{\#s266,N,A,MI\} where MI<:Tuple\{Vararg\{Base.MultiplicativeInverses.SignedMultiplicativeInverse\{Int64\},N\} where N\} where A<:Union\{DenseArray, SubArray\{T,N,P,I,true\} where I<:Tuple\{Union\{Base.Slice, UnitRange\},Vararg\{Any,N\} where N\} where P where N where T\}, DenseArray\{\#s266,N\}, SubArray\{\#s266,N,A,I,L\} where L\} where I<:Tuple\{Vararg\{Union\{Base.AbstractCartesianIndex, Int64, Range\{Int64\}\},N\} where N\} where A<:Union\{Base.ReshapedArray\{T,N,A,MI\} where MI<:Tuple\{Vararg\{Base.MultiplicativeInverses.SignedMultiplicativeInverse\{Int64\},N\} where N\} where A<:Union\{DenseArray, SubArray\{T,N,P,I,true\} where I<:Tuple\{Union\{Base.Slice, UnitRange\},Vararg\{Any,N\} where N\} where P where N where T\} where N where T, DenseArray\} where N where \#s266<:Union\{Base.Dates.CompoundPeriod, Base.Dates.Period\}, Y::Union\{Base.ReshapedArray\{\#s265,N,A,MI\} where MI<:Tuple\{Vararg\{Base.MultiplicativeInverses.SignedMultiplicativeInverse\{Int64\},N\} where N\} where A<:Union\{DenseArray, SubArray\{T,N,P,I,true\} where I<:Tuple\{Union\{Base.Slice, UnitRange\},Vararg\{Any,N\} where N\} where P where N where T\}, DenseArray\{\#s265,N\}, SubArray\{\#s265,N,A,I,L\} where L\} where I<:Tuple\{Vararg\{Union\{Base.AbstractCartesianIndex, Int64, Range\{Int64\}\},N\} where N\} where A<:Union\{Base.ReshapedArray\{T,N,A,MI\} where MI<:Tuple\{Vararg\{Base.MultiplicativeInverses.SignedMultiplicativeInverse\{Int64\},N\} where N\} where A<:Union\{DenseArray, SubArray\{T,N,P,I,true\} where I<:Tuple\{Union\{Base.Slice, UnitRange\},Vararg\{Any,N\} where N\} where P where N where T\} where N where T, DenseArray\} where N where \#s265<:Union\{Base.Dates.CompoundPeriod, Base.Dates.Period\}) in Base.Dates at dates/periods.jl:364
         +(x::Union\{Base.ReshapedArray\{\#s268,N,A,MI\} where MI<:Tuple\{Vararg\{Base.MultiplicativeInverses.SignedMultiplicativeInverse\{Int64\},N\} where N\} where A<:Union\{DenseArray, SubArray\{T,N,P,I,true\} where I<:Tuple\{Union\{Base.Slice, UnitRange\},Vararg\{Any,N\} where N\} where P where N where T\}, DenseArray\{\#s268,N\}, SubArray\{\#s268,N,A,I,L\} where L\} where I<:Tuple\{Vararg\{Union\{Base.AbstractCartesianIndex, Int64, Range\{Int64\}\},N\} where N\} where A<:Union\{Base.ReshapedArray\{T,N,A,MI\} where MI<:Tuple\{Vararg\{Base.MultiplicativeInverses.SignedMultiplicativeInverse\{Int64\},N\} where N\} where A<:Union\{DenseArray, SubArray\{T,N,P,I,true\} where I<:Tuple\{Union\{Base.Slice, UnitRange\},Vararg\{Any,N\} where N\} where P where N where T\} where N where T, DenseArray\} where N where \#s268<:Union\{Base.Dates.CompoundPeriod, Base.Dates.Period\}, y::Base.Dates.TimeType) in Base.Dates at dates/arithmetic.jl:86
         +(r::Range\{\#s268\} where \#s268<:Base.Dates.TimeType, x::Base.Dates.Period) in Base.Dates at dates/ranges.jl:47
         +(A::SparseMatrixCSC, B::SparseMatrixCSC) in Base.SparseArrays at sparse/sparsematrix.jl:1537
         +(A::SparseMatrixCSC, B::Array) in Base.SparseArrays at sparse/sparsematrix.jl:1540
         +(x::AbstractSparseArray\{Tv,Ti,1\} where Ti where Tv, y::AbstractSparseArray\{Tv,Ti,1\} where Ti where Tv) in Base.SparseArrays at sparse/sparsevector.jl:1339
         +(x::AbstractSparseArray\{Tv,Ti,1\} where Ti where Tv, y::Union\{Base.ReshapedArray\{T,1,A,MI\} where MI<:Tuple\{Vararg\{Base.MultiplicativeInverses.SignedMultiplicativeInverse\{Int64\},N\} where N\} where A<:Union\{DenseArray, SubArray\{T,N,P,I,true\} where I<:Tuple\{Union\{Base.Slice, UnitRange\},Vararg\{Any,N\} where N\} where P where N where T\}, DenseArray\{T,1\}, SubArray\{T,1,A,I,L\} where L\} where I<:Tuple\{Vararg\{Union\{Base.AbstractCartesianIndex, Int64, Range\{Int64\}\},N\} where N\} where A<:Union\{Base.ReshapedArray\{T,N,A,MI\} where MI<:Tuple\{Vararg\{Base.MultiplicativeInverses.SignedMultiplicativeInverse\{Int64\},N\} where N\} where A<:Union\{DenseArray, SubArray\{T,N,P,I,true\} where I<:Tuple\{Union\{Base.Slice, UnitRange\},Vararg\{Any,N\} where N\} where P where N where T\} where N where T, DenseArray\} where T) in Base.SparseArrays at sparse/sparsevector.jl:1341
         +(x::AbstractArray\{\#s45,N\} where N where \#s45<:Number) in Base at abstractarraymath.jl:95
         +(A::AbstractArray, B::AbstractArray) in Base at arraymath.jl:38
         +(A::Number, B::AbstractArray) in Base at arraymath.jl:45
         +(A::AbstractArray, B::Number) in Base at arraymath.jl:48
         +(index1::CartesianIndex\{N\}, index2::CartesianIndex\{N\}) where N in Base.IteratorsMD at multidimensional.jl:101
         +(index::CartesianIndex\{N\}, i::Integer) where N in Base.IteratorsMD at multidimensional.jl:111
         +(J1::UniformScaling, J2::UniformScaling) in Base.LinAlg at linalg/uniformscaling.jl:58
         +(J::UniformScaling, B::BitArray\{2\}) in Base.LinAlg at linalg/uniformscaling.jl:60
         +(J::UniformScaling, A::AbstractArray\{T,2\} where T) in Base.LinAlg at linalg/uniformscaling.jl:61
         +(a::Base.Pkg.Resolve.VersionWeights.HierarchicalValue\{T\}, b::Base.Pkg.Resolve.VersionWeights.HierarchicalValue\{T\}) where T in Base.Pkg.Resolve.VersionWeights at pkg/resolve/versionweight.jl:23
         +(x::P, y::P) where P<:Base.Dates.Period in Base.Dates at dates/periods.jl:70
         +(x::Base.Dates.Period, y::Base.Dates.Period) in Base.Dates at dates/periods.jl:346
         +(y::Base.Dates.Period, x::Base.Dates.CompoundPeriod) in Base.Dates at dates/periods.jl:348
         +(x::Union\{Base.Dates.CompoundPeriod, Base.Dates.Period\}) in Base.Dates at dates/periods.jl:357
         +(x::Union\{Base.Dates.CompoundPeriod, Base.Dates.Period\}, Y::Union\{Base.ReshapedArray\{\#s268,N,A,MI\} where MI<:Tuple\{Vararg\{Base.MultiplicativeInverses.SignedMultiplicativeInverse\{Int64\},N\} where N\} where A<:Union\{DenseArray, SubArray\{T,N,P,I,true\} where I<:Tuple\{Union\{Base.Slice, UnitRange\},Vararg\{Any,N\} where N\} where P where N where T\}, DenseArray\{\#s268,N\}, SubArray\{\#s268,N,A,I,L\} where L\} where I<:Tuple\{Vararg\{Union\{Base.AbstractCartesianIndex, Int64, Range\{Int64\}\},N\} where N\} where A<:Union\{Base.ReshapedArray\{T,N,A,MI\} where MI<:Tuple\{Vararg\{Base.MultiplicativeInverses.SignedMultiplicativeInverse\{Int64\},N\} where N\} where A<:Union\{DenseArray, SubArray\{T,N,P,I,true\} where I<:Tuple\{Union\{Base.Slice, UnitRange\},Vararg\{Any,N\} where N\} where P where N where T\} where N where T, DenseArray\} where N where \#s268<:Union\{Base.Dates.CompoundPeriod, Base.Dates.Period\}) in Base.Dates at dates/periods.jl:362
         +(x::Base.Dates.TimeType) in Base.Dates at dates/arithmetic.jl:8
         +(a::Base.Dates.TimeType, b::Base.Dates.Period, c::Base.Dates.Period) in Base.Dates at dates/periods.jl:378
         +(a::Base.Dates.TimeType, b::Base.Dates.Period, c::Base.Dates.Period, d::Base.Dates.Period{\ldots}) in Base.Dates at dates/periods.jl:379
         +(x::Base.Dates.TimeType, y::Base.Dates.CompoundPeriod) in Base.Dates at dates/periods.jl:382
         +(x::Base.Dates.Instant) in Base.Dates at dates/arithmetic.jl:4
         +(y::Base.Dates.Period, x::Base.Dates.TimeType) in Base.Dates at dates/arithmetic.jl:83
         +(x::AbstractArray\{\#s268,N\} where N where \#s268<:Base.Dates.TimeType, y::Union\{Base.Dates.CompoundPeriod, Base.Dates.Period\}) in Base.Dates at dates/arithmetic.jl:85
         +(x::Base.Dates.Period, r::Range\{\#s268\} where \#s268<:Base.Dates.TimeType) in Base.Dates at dates/ranges.jl:46
         +(y::Union\{Base.Dates.CompoundPeriod, Base.Dates.Period\}, x::AbstractArray\{\#s268,N\} where N where \#s268<:Base.Dates.TimeType) in Base.Dates at dates/arithmetic.jl:87
         +(y::Base.Dates.TimeType, x::Union\{Base.ReshapedArray\{\#s268,N,A,MI\} where MI<:Tuple\{Vararg\{Base.MultiplicativeInverses.SignedMultiplicativeInverse\{Int64\},N\} where N\} where A<:Union\{DenseArray, SubArray\{T,N,P,I,true\} where I<:Tuple\{Union\{Base.Slice, UnitRange\},Vararg\{Any,N\} where N\} where P where N where T\}, DenseArray\{\#s268,N\}, SubArray\{\#s268,N,A,I,L\} where L\} where I<:Tuple\{Vararg\{Union\{Base.AbstractCartesianIndex, Int64, Range\{Int64\}\},N\} where N\} where A<:Union\{Base.ReshapedArray\{T,N,A,MI\} where MI<:Tuple\{Vararg\{Base.MultiplicativeInverses.SignedMultiplicativeInverse\{Int64\},N\} where N\} where A<:Union\{DenseArray, SubArray\{T,N,P,I,true\} where I<:Tuple\{Union\{Base.Slice, UnitRange\},Vararg\{Any,N\} where N\} where P where N where T\} where N where T, DenseArray\} where N where \#s268<:Union\{Base.Dates.CompoundPeriod, Base.Dates.Period\}) in Base.Dates at dates/arithmetic.jl:88
         +(J::UniformScaling, x::Number) in Base at deprecated.jl:56
         +(x::Number, J::UniformScaling) in Base at deprecated.jl:56
         +(a, b, c, xs{\ldots}) in Base at operators.jl:424
\end{Verbatim}
            
    \begin{Verbatim}[commandchars=\\\{\}]
{\color{incolor}In [{\color{incolor}19}]:} \PY{c}{\PYZsh{}Podemos definir mas metodos. Para esto primero tenemos que importar + de Base}
         \PY{k}{import} \PY{n}{Base}\PY{o}{:} \PY{o}{+}
\end{Verbatim}


    \begin{Verbatim}[commandchars=\\\{\}]
{\color{incolor}In [{\color{incolor}20}]:} \PY{c}{\PYZsh{}Por ejemplo si queremos concatenar elementos con +. Sin extender el metodo, no funciona. (No esta entre esos 180 anteriores)}
         \PY{o}{+}\PY{p}{(}\PY{n}{x}\PY{o}{::}\PY{n}{String}\PY{p}{,} \PY{n}{y}\PY{o}{::}\PY{n}{String}\PY{p}{)} \PY{o}{=} \PY{n}{string}\PY{p}{(}\PY{n}{x}\PY{p}{,} \PY{n}{y}\PY{p}{)}
\end{Verbatim}


\begin{Verbatim}[commandchars=\\\{\}]
{\color{outcolor}Out[{\color{outcolor}20}]:} + (generic function with 181 methods)
\end{Verbatim}
            
    \begin{Verbatim}[commandchars=\\\{\}]
{\color{incolor}In [{\color{incolor}21}]:} \PY{l+s}{\PYZdq{}}\PY{l+s}{H}\PY{l+s}{e}\PY{l+s}{l}\PY{l+s}{l}\PY{l+s}{o}\PY{l+s}{\PYZdq{}} \PY{o}{+} \PY{l+s}{\PYZdq{}}\PY{l+s}{ }\PY{l+s}{w}\PY{l+s}{o}\PY{l+s}{r}\PY{l+s}{l}\PY{l+s}{d}\PY{l+s}{!}\PY{l+s}{\PYZdq{}}
\end{Verbatim}


\begin{Verbatim}[commandchars=\\\{\}]
{\color{outcolor}Out[{\color{outcolor}21}]:} "Helloworld!"
\end{Verbatim}
            
    \begin{Verbatim}[commandchars=\\\{\}]
{\color{incolor}In [{\color{incolor}22}]:} \PY{c}{\PYZsh{}Un ejemplo mas complejo}
         \PY{n}{foo}\PY{p}{(}\PY{n}{x}\PY{p}{,} \PY{n}{y}\PY{p}{)} \PY{o}{=} \PY{n}{println}\PY{p}{(}\PY{l+s}{\PYZdq{}}\PY{l+s}{d}\PY{l+s}{u}\PY{l+s}{c}\PY{l+s}{k}\PY{l+s}{\PYZhy{}}\PY{l+s}{t}\PY{l+s}{y}\PY{l+s}{p}\PY{l+s}{e}\PY{l+s}{d}\PY{l+s}{ }\PY{l+s}{f}\PY{l+s}{o}\PY{l+s}{o}\PY{l+s}{!}\PY{l+s}{\PYZdq{}}\PY{p}{)}
         \PY{n}{foo}\PY{p}{(}\PY{n}{x}\PY{o}{::}\PY{k+kt}{Int}\PY{p}{,} \PY{n}{y}\PY{o}{::}\PY{k+kt}{Float64}\PY{p}{)} \PY{o}{=} \PY{n}{println}\PY{p}{(}\PY{l+s}{\PYZdq{}}\PY{l+s}{f}\PY{l+s}{o}\PY{l+s}{o}\PY{l+s}{ }\PY{l+s}{c}\PY{l+s}{o}\PY{l+s}{n}\PY{l+s}{ }\PY{l+s}{e}\PY{l+s}{n}\PY{l+s}{t}\PY{l+s}{e}\PY{l+s}{r}\PY{l+s}{o}\PY{l+s}{ }\PY{l+s}{y}\PY{l+s}{ }\PY{l+s}{f}\PY{l+s}{l}\PY{l+s}{o}\PY{l+s}{t}\PY{l+s}{a}\PY{l+s}{n}\PY{l+s}{t}\PY{l+s}{e}\PY{l+s}{!}\PY{l+s}{\PYZdq{}}\PY{p}{)}
         \PY{n}{foo}\PY{p}{(}\PY{n}{x}\PY{o}{::}\PY{k+kt}{Float64}\PY{p}{,} \PY{n}{y}\PY{o}{::}\PY{k+kt}{Float64}\PY{p}{)} \PY{o}{=} \PY{n}{println}\PY{p}{(}\PY{l+s}{\PYZdq{}}\PY{l+s}{f}\PY{l+s}{o}\PY{l+s}{o}\PY{l+s}{ }\PY{l+s}{c}\PY{l+s}{o}\PY{l+s}{n}\PY{l+s}{ }\PY{l+s}{d}\PY{l+s}{o}\PY{l+s}{s}\PY{l+s}{ }\PY{l+s}{f}\PY{l+s}{l}\PY{l+s}{o}\PY{l+s}{t}\PY{l+s}{a}\PY{l+s}{n}\PY{l+s}{t}\PY{l+s}{e}\PY{l+s}{s}\PY{l+s}{!}\PY{l+s}{\PYZdq{}}\PY{p}{)}
         \PY{n}{foo}\PY{p}{(}\PY{n}{x}\PY{o}{::}\PY{k+kt}{Int}\PY{p}{,} \PY{n}{y}\PY{o}{::}\PY{k+kt}{Int}\PY{p}{)} \PY{o}{=} \PY{n}{println}\PY{p}{(}\PY{l+s}{\PYZdq{}}\PY{l+s}{f}\PY{l+s}{o}\PY{l+s}{o}\PY{l+s}{ }\PY{l+s}{c}\PY{l+s}{o}\PY{l+s}{n}\PY{l+s}{ }\PY{l+s}{d}\PY{l+s}{o}\PY{l+s}{s}\PY{l+s}{ }\PY{l+s}{e}\PY{l+s}{n}\PY{l+s}{t}\PY{l+s}{e}\PY{l+s}{r}\PY{l+s}{o}\PY{l+s}{s}\PY{l+s}{\PYZdq{}}\PY{p}{)}
\end{Verbatim}


\begin{Verbatim}[commandchars=\\\{\}]
{\color{outcolor}Out[{\color{outcolor}22}]:} foo (generic function with 4 methods)
\end{Verbatim}
            
    \begin{Verbatim}[commandchars=\\\{\}]
{\color{incolor}In [{\color{incolor}23}]:} \PY{n}{foo}\PY{p}{(}\PY{l+m+mi}{1}\PY{p}{,} \PY{l+m+mi}{1}\PY{p}{)}
\end{Verbatim}


    \begin{Verbatim}[commandchars=\\\{\}]
foo con dos enteros

    \end{Verbatim}

    \begin{Verbatim}[commandchars=\\\{\}]
{\color{incolor}In [{\color{incolor}24}]:} \PY{n}{foo}\PY{p}{(}\PY{l+m+mf}{1.} \PY{p}{,} \PY{l+m+mf}{1.}\PY{p}{)}
\end{Verbatim}


    \begin{Verbatim}[commandchars=\\\{\}]
foo con dos flotantes!

    \end{Verbatim}

    \begin{Verbatim}[commandchars=\\\{\}]
{\color{incolor}In [{\color{incolor}25}]:} \PY{n}{foo}\PY{p}{(}\PY{l+m+mi}{1}\PY{p}{,} \PY{l+m+mf}{1.0}\PY{p}{)}
\end{Verbatim}


    \begin{Verbatim}[commandchars=\\\{\}]
foo con entero y flotante!

    \end{Verbatim}

    \begin{Verbatim}[commandchars=\\\{\}]
{\color{incolor}In [{\color{incolor}26}]:} \PY{n}{foo}\PY{p}{(}\PY{k+kc}{true}\PY{p}{,} \PY{k+kc}{false}\PY{p}{)}
\end{Verbatim}


    \begin{Verbatim}[commandchars=\\\{\}]
duck-typed foo!

    \end{Verbatim}

    \subsection{Paradigmas que soporta}\label{paradigmas-que-soporta}

\subsubsection{Funcional}\label{funcional}

    \begin{Verbatim}[commandchars=\\\{\}]
{\color{incolor}In [{\color{incolor}17}]:} \PY{n}{map}\PY{p}{(}\PY{n}{x} \PY{o}{\PYZhy{}}\PY{o}{\PYZgt{}} \PY{n}{x}\PY{o}{\PYZca{}}\PY{l+m+mi}{3}\PY{p}{,} \PY{p}{(}\PY{l+m+mi}{1}\PY{o}{:}\PY{l+m+mi}{5}\PY{p}{)}\PY{p}{)}
\end{Verbatim}


\begin{Verbatim}[commandchars=\\\{\}]
{\color{outcolor}Out[{\color{outcolor}17}]:} 5-element Array\{Int64,1\}:
            1
            8
           27
           64
          125
\end{Verbatim}
            
    \begin{Verbatim}[commandchars=\\\{\}]
{\color{incolor}In [{\color{incolor}5}]:} \PY{n}{filter}\PY{p}{(}\PY{n}{isprime}\PY{p}{,} \PY{p}{(}\PY{l+m+mi}{1}\PY{o}{:}\PY{l+m+mi}{50}\PY{p}{)}\PY{p}{)}
\end{Verbatim}


\begin{Verbatim}[commandchars=\\\{\}]
{\color{outcolor}Out[{\color{outcolor}5}]:} 15-element Array\{Int64,1\}:
          2
          3
          5
          7
         11
         13
         17
         19
         23
         29
         31
         37
         41
         43
         47
\end{Verbatim}
            
    \subsubsection{Orientado a objetos}\label{orientado-a-objetos}

    \begin{Verbatim}[commandchars=\\\{\}]
{\color{incolor}In [{\color{incolor}18}]:} \PY{k}{type} \PY{n}{Tigre}
           \PY{n}{largo\PYZus{}de\PYZus{}cola}\PY{o}{::}\PY{k+kt}{Float64}
           \PY{n}{color\PYZus{}de\PYZus{}pelaje}
         \PY{k}{end}
\end{Verbatim}


    \begin{Verbatim}[commandchars=\\\{\}]
{\color{incolor}In [{\color{incolor}19}]:} \PY{k}{abstract} \PY{n}{Felino} \PY{c}{\PYZsh{} Declaramos una clase abstracta sin comportamiento}
\end{Verbatim}


    \begin{Verbatim}[commandchars=\\\{\}]
{\color{incolor}In [{\color{incolor}20}]:} \PY{k}{type} \PY{n}{Pantera} \PY{o}{\PYZlt{}:} \PY{n}{Felino} \PY{c}{\PYZsh{} Panther is also a subtype of Cat}
             \PY{n}{color\PYZus{}de\PYZus{}ojos}
             \PY{n}{color\PYZus{}de\PYZus{}pelaje}
           \PY{n}{Pantera}\PY{p}{(}\PY{p}{)} \PY{o}{=} \PY{n}{new}\PY{p}{(}\PY{l+s}{\PYZdq{}}\PY{l+s}{v}\PY{l+s}{e}\PY{l+s}{r}\PY{l+s}{d}\PY{l+s}{e}\PY{l+s}{\PYZdq{}}\PY{p}{,} \PY{l+s}{\PYZdq{}}\PY{l+s}{n}\PY{l+s}{e}\PY{l+s}{g}\PY{l+s}{r}\PY{l+s}{o}\PY{l+s}{\PYZdq{}}\PY{p}{)}
         \PY{k}{end}
\end{Verbatim}


    \begin{Verbatim}[commandchars=\\\{\}]
{\color{incolor}In [{\color{incolor}21}]:} \PY{k}{type} \PY{n}{Leon} \PY{o}{\PYZlt{}:} \PY{n}{Felino} \PY{c}{\PYZsh{} Lion is a subtype of Cat}
           \PY{n}{color\PYZus{}de\PYZus{}melena}
           \PY{n}{roar}\PY{o}{::}\PY{k+kt}{AbstractString}
         \PY{k}{end}
\end{Verbatim}


    \begin{Verbatim}[commandchars=\\\{\}]
{\color{incolor}In [{\color{incolor}22}]:} \PY{n}{tigger} \PY{o}{=} \PY{n}{Tigre}\PY{p}{(}\PY{l+m+mf}{3.5}\PY{p}{,}\PY{l+s}{\PYZdq{}}\PY{l+s}{n}\PY{l+s}{a}\PY{l+s}{r}\PY{l+s}{a}\PY{l+s}{n}\PY{l+s}{j}\PY{l+s}{a}\PY{l+s}{\PYZdq{}}\PY{p}{)}
\end{Verbatim}


\begin{Verbatim}[commandchars=\\\{\}]
{\color{outcolor}Out[{\color{outcolor}22}]:} Tigre(3.5,"naranja")
\end{Verbatim}
            
    \begin{Verbatim}[commandchars=\\\{\}]
{\color{incolor}In [{\color{incolor}23}]:} \PY{k}{function} \PY{n}{meow}\PY{p}{(}\PY{n}{animal}\PY{o}{::}\PY{n}{Leon}\PY{p}{)}
           \PY{n}{animal}\PY{o}{.}\PY{n}{roar}
         \PY{k}{end}
         
         \PY{k}{function} \PY{n}{meow}\PY{p}{(}\PY{n}{animal}\PY{o}{::}\PY{n}{Pantera}\PY{p}{)}
           \PY{l+s}{\PYZdq{}}\PY{l+s}{g}\PY{l+s}{r}\PY{l+s}{r}\PY{l+s}{r}\PY{l+s}{\PYZdq{}}
         \PY{k}{end}
         
         \PY{k}{function} \PY{n}{meow}\PY{p}{(}\PY{n}{animal}\PY{o}{::}\PY{n}{Tigre}\PY{p}{)}
           \PY{l+s}{\PYZdq{}}\PY{l+s}{r}\PY{l+s}{a}\PY{l+s}{w}\PY{l+s}{w}\PY{l+s}{w}\PY{l+s}{r}\PY{l+s}{\PYZdq{}}
         \PY{k}{end}
\end{Verbatim}


\begin{Verbatim}[commandchars=\\\{\}]
{\color{outcolor}Out[{\color{outcolor}23}]:} meow (generic function with 3 methods)
\end{Verbatim}
            
    \begin{Verbatim}[commandchars=\\\{\}]
{\color{incolor}In [{\color{incolor}24}]:} \PY{n}{meow}\PY{p}{(}\PY{n}{tigger}\PY{p}{)}
\end{Verbatim}


\begin{Verbatim}[commandchars=\\\{\}]
{\color{outcolor}Out[{\color{outcolor}24}]:} "rawwwr"
\end{Verbatim}
            
    \begin{Verbatim}[commandchars=\\\{\}]
{\color{incolor}In [{\color{incolor}25}]:} \PY{n}{meow}\PY{p}{(}\PY{n}{Leon}\PY{p}{(}\PY{l+s}{\PYZdq{}}\PY{l+s}{m}\PY{l+s}{a}\PY{l+s}{r}\PY{l+s}{r}\PY{l+s}{ó}\PY{l+s}{n}\PY{l+s}{\PYZdq{}}\PY{p}{,}\PY{l+s}{\PYZdq{}}\PY{l+s}{R}\PY{l+s}{O}\PY{l+s}{A}\PY{l+s}{A}\PY{l+s}{R}\PY{l+s}{\PYZdq{}}\PY{p}{)}\PY{p}{)}
\end{Verbatim}


\begin{Verbatim}[commandchars=\\\{\}]
{\color{outcolor}Out[{\color{outcolor}25}]:} "ROAAR"
\end{Verbatim}
            
    \begin{Verbatim}[commandchars=\\\{\}]
{\color{incolor}In [{\color{incolor}26}]:} \PY{n}{meow}\PY{p}{(}\PY{n}{Pantera}\PY{p}{(}\PY{p}{)}\PY{p}{)}
\end{Verbatim}


\begin{Verbatim}[commandchars=\\\{\}]
{\color{outcolor}Out[{\color{outcolor}26}]:} "grrr"
\end{Verbatim}
            
    \section{Caracteristicas avanzadas del
lenguaje}\label{caracteristicas-avanzadas-del-lenguaje}

    Diseñado para paralelismo y computción distribuida

Corutinas también llamadas: Lightweight threading

Macros parecidos a los de Lisp y otras facilidades para metaprogramacion

Acepta multiple dispach

    \section{Paralelismo}\label{paralelismo}

    \begin{itemize}
\item
  El paralelismo en Julia se hace por CPU, no GPU.
\item
  Los procesos se llaman Workers con un ID específico.
\end{itemize}

    \begin{Verbatim}[commandchars=\\\{\}]
{\color{incolor}In [{\color{incolor}21}]:} \PY{n}{println}\PY{p}{(}\PY{n}{nprocs}\PY{p}{(}\PY{p}{)}\PY{p}{)}
         \PY{n}{addprocs}\PY{p}{(}\PY{l+m+mi}{1}\PY{p}{)}
         \PY{n}{println}\PY{p}{(}\PY{n}{nprocs}\PY{p}{(}\PY{p}{)}\PY{p}{)}
\end{Verbatim}


    \begin{Verbatim}[commandchars=\\\{\}]
1
2

    \end{Verbatim}

    \begin{Verbatim}[commandchars=\\\{\}]
{\color{incolor}In [{\color{incolor}1}]:} \PY{k}{for} \PY{n}{w} \PY{k+kp}{in} \PY{n}{workers}\PY{p}{(}\PY{p}{)}
            \PY{n}{rref}\PY{o}{=}\PY{n}{remotecall}\PY{p}{(}\PY{n}{myid}\PY{p}{,} \PY{n}{w}\PY{p}{)}
            \PY{n}{sleep}\PY{p}{(}\PY{l+m+mi}{1}\PY{p}{)}
            \PY{n}{println}\PY{p}{(}\PY{n}{fetch}\PY{p}{(}\PY{n}{rref}\PY{p}{)}\PY{p}{)}
        \PY{k}{end}
\end{Verbatim}


    \begin{Verbatim}[commandchars=\\\{\}]
1

    \end{Verbatim}

    \begin{Verbatim}[commandchars=\\\{\}]
{\color{incolor}In [{\color{incolor}29}]:} \PY{n+nd}{@everywhere} \PY{k}{function} \PY{n}{random\PYZus{}num}\PY{p}{(}\PY{n}{n}\PY{p}{)}
             \PY{n}{c}\PY{o}{::}\PY{k+kt}{Int} \PY{o}{=} \PY{l+m+mi}{0}
             \PY{k}{for} \PY{n}{i} \PY{o}{=} \PY{l+m+mi}{1}\PY{o}{:}\PY{n}{n}
                 \PY{n}{c} \PY{o}{+=} \PY{n}{rand}\PY{p}{(}\PY{k+kt}{Bool}\PY{p}{)}
             \PY{k}{end}
             \PY{n}{c}
         \PY{k}{end}
         
         \PY{n}{a} \PY{o}{=} \PY{n+nd}{@spawn} \PY{n}{random\PYZus{}num}\PY{p}{(}\PY{l+m+mi}{100000000}\PY{p}{)}
         
         \PY{n}{b} \PY{o}{=} \PY{n+nd}{@spawn} \PY{n}{random\PYZus{}num}\PY{p}{(}\PY{l+m+mi}{10000000}\PY{p}{)}
         
         \PY{n}{println}\PY{p}{(}\PY{n}{fetch}\PY{p}{(}\PY{n}{a}\PY{p}{)}\PY{o}{+}\PY{n}{fetch}\PY{p}{(}\PY{n}{b}\PY{p}{)}\PY{p}{)} \PY{c}{\PYZsh{}reducción}
\end{Verbatim}


    \begin{Verbatim}[commandchars=\\\{\}]
55002896

    \end{Verbatim}

    \begin{Verbatim}[commandchars=\\\{\}]
{\color{incolor}In [{\color{incolor}33}]:} \PY{c}{\PYZsh{} Se genera la lista de tasks}
         \PY{n}{t} \PY{o}{=} \PY{n+nd}{@task} \PY{k+kt}{Any}\PY{p}{[} \PY{k}{for} \PY{n}{x} \PY{k+kp}{in} \PY{p}{[}\PY{l+m+mi}{1}\PY{p}{,}\PY{l+m+mi}{2}\PY{p}{,}\PY{l+m+mi}{4}\PY{p}{]} \PY{n}{println}\PY{p}{(}\PY{n}{x}\PY{p}{)} \PY{k}{end} \PY{p}{]}
         
         \PY{c}{\PYZsh{} No hay tasks?}
         \PY{n}{println}\PY{p}{(}\PY{n}{istaskdone}\PY{p}{(}\PY{n}{t}\PY{p}{)}\PY{p}{)}
         
         \PY{c}{\PYZsh{} Cual es la siguiente task?}
         \PY{n}{println}\PY{p}{(}\PY{n}{current\PYZus{}task}\PY{p}{(}\PY{p}{)}\PY{p}{)}
         
         \PY{c}{\PYZsh{} Ejecutar task}
         \PY{n}{println}\PY{p}{(}\PY{n}{consume}\PY{p}{(}\PY{n}{t}\PY{p}{)}\PY{p}{)}
\end{Verbatim}


    \begin{Verbatim}[commandchars=\\\{\}]
false
Task (runnable) @0x00007fb443677490
1
2
4
Any[nothing]

    \end{Verbatim}

    \begin{Verbatim}[commandchars=\\\{\}]
{\color{incolor}In [{\color{incolor} }]:} \PY{n}{c1} \PY{o}{=} \PY{k+kt}{Channel}\PY{p}{(}\PY{l+m+mi}{32}\PY{p}{)}
        \PY{n}{put!}\PY{p}{(}\PY{n}{c1}\PY{p}{,}\PY{l+m+mi}{3}\PY{p}{)}
        \PY{n}{put!}\PY{p}{(}\PY{n}{c1}\PY{p}{,}\PY{l+m+mi}{4}\PY{p}{)}
        \PY{n}{c2} \PY{o}{=} \PY{k+kt}{Channel}\PY{p}{(}\PY{l+m+mi}{32}\PY{p}{)}
        
        \PY{c}{\PYZsh{} foo() lee un item de c1, lo imprime y lo escribe en c2}
        \PY{k}{function} \PY{n}{foo}\PY{p}{(}\PY{p}{)}
            \PY{k}{while} \PY{k+kc}{true}
                \PY{n}{data} \PY{o}{=} \PY{n}{take!}\PY{p}{(}\PY{n}{c1}\PY{p}{)}
                \PY{n}{println}\PY{p}{(}\PY{n}{data}\PY{p}{)}
                \PY{n}{result} \PY{o}{=} \PY{n}{data} \PY{o}{+} \PY{l+m+mi}{2}
                \PY{n}{put!}\PY{p}{(}\PY{n}{c2}\PY{p}{,} \PY{n}{result}\PY{p}{)}   
            \PY{k}{end}
        \PY{k}{end}
        
        \PY{c}{\PYZsh{} con @schedule podemos hacer que varias instancias de foo() estén activas concurrentemente.}
        \PY{k}{for} \PY{n}{i} \PY{k+kp}{in} \PY{l+m+mi}{1}\PY{o}{:}\PY{l+m+mi}{3}
            \PY{n+nd}{@schedule} \PY{n}{foo}\PY{p}{(}\PY{p}{)}
        \PY{k}{end}
        
        \PY{k}{for} \PY{n}{i} \PY{k+kp}{in} \PY{l+m+mi}{1}\PY{o}{:}\PY{l+m+mi}{3}
            \PY{n}{data}\PY{o}{=} \PY{n}{take!}\PY{p}{(}\PY{n}{c2}\PY{p}{)}
            \PY{n}{println}\PY{p}{(}\PY{n}{data}\PY{p}{)}
        \PY{k}{end}
\end{Verbatim}


    \begin{Verbatim}[commandchars=\\\{\}]
3
4
5
6

    \end{Verbatim}

    \begin{Verbatim}[commandchars=\\\{\}]
{\color{incolor}In [{\color{incolor}1}]:} \PY{n}{c} \PY{o}{=} \PY{k+kt}{Channel}\PY{p}{(}\PY{l+m+mi}{0}\PY{p}{)}
        
        \PY{n}{task} \PY{o}{=} \PY{n+nd}{@schedule} \PY{n}{foreach}\PY{p}{(}\PY{n}{i}\PY{o}{\PYZhy{}}\PY{o}{\PYZgt{}}\PY{n}{put!}\PY{p}{(}\PY{n}{c}\PY{p}{,} \PY{n}{i}\PY{p}{)}\PY{p}{,} \PY{l+m+mi}{1}\PY{o}{:}\PY{l+m+mi}{4}\PY{p}{)}
        
        \PY{n}{bind}\PY{p}{(}\PY{n}{c}\PY{p}{,}\PY{n}{task}\PY{p}{)}
        
        \PY{k}{for} \PY{n}{i} \PY{k+kp}{in} \PY{n}{c}
            \PY{n+nd}{@show} \PY{n}{i}
        \PY{k}{end}
        
        \PY{n}{isopen}\PY{p}{(}\PY{n}{c}\PY{p}{)}
\end{Verbatim}


    \begin{Verbatim}[commandchars=\\\{\}]
i = 1
i = 2
i = 3
i = 4

    \end{Verbatim}

\begin{Verbatim}[commandchars=\\\{\}]
{\color{outcolor}Out[{\color{outcolor}1}]:} false
\end{Verbatim}
            
    \subsubsection{Machine Learning}\label{machine-learning}

\paragraph{KMeans: se inicializan k-centroides y se asigna a cada punto
del set de datos el centroide más cercano. Se recalculan los centroides
como promedio de todos los puntos y se repite hasta lograr la
convergencia.}\label{kmeans-se-inicializan-k-centroides-y-se-asigna-a-cada-punto-del-set-de-datos-el-centroide-muxe1s-cercano.-se-recalculan-los-centroides-como-promedio-de-todos-los-puntos-y-se-repite-hasta-lograr-la-convergencia.}

    \begin{Verbatim}[commandchars=\\\{\}]
{\color{incolor}In [{\color{incolor} }]:} \PY{k}{using} \PY{n}{RDatasets}
        \PY{n}{xclara} \PY{o}{=} \PY{n}{dataset}\PY{p}{(}\PY{l+s}{\PYZdq{}}\PY{l+s}{c}\PY{l+s}{l}\PY{l+s}{u}\PY{l+s}{s}\PY{l+s}{t}\PY{l+s}{e}\PY{l+s}{r}\PY{l+s}{\PYZdq{}}\PY{p}{,} \PY{l+s}{\PYZdq{}}\PY{l+s}{x}\PY{l+s}{c}\PY{l+s}{l}\PY{l+s}{a}\PY{l+s}{r}\PY{l+s}{a}\PY{l+s}{\PYZdq{}}\PY{p}{)}
        \PY{n}{size}\PY{p}{(}\PY{n}{xclara}\PY{p}{)}
\end{Verbatim}


    \begin{Verbatim}[commandchars=\\\{\}]
{\color{incolor}In [{\color{incolor} }]:} \PY{c}{\PYZsh{} Aca tambien es necesario tener plotly: pip install plotly}
        \PY{c}{\PYZsh{} y Pkg.add(\PYZdq{}PyPlot\PYZdq{})}
        
        \PY{n}{x} \PY{o}{=} \PY{n}{xclara}\PY{p}{[}\PY{o}{:}\PY{n}{V1}\PY{p}{]}
        \PY{n}{y} \PY{o}{=} \PY{n}{xclara}\PY{p}{[}\PY{o}{:}\PY{n}{V2}\PY{p}{]}
        \PY{k}{using} \PY{n}{Plots}
        \PY{n}{scatter}\PY{p}{(}\PY{n}{x}\PY{p}{,} \PY{n}{y} \PY{p}{,}\PY{n}{alpha}\PY{o}{=}\PY{l+m+mf}{0.5}\PY{p}{)}
\end{Verbatim}


    \begin{Verbatim}[commandchars=\\\{\}]
{\color{incolor}In [{\color{incolor} }]:} \PY{k}{using} \PY{n}{Clustering}
        
        \PY{n}{xclara} \PY{o}{=} \PY{n}{convert}\PY{p}{(}\PY{k+kt}{Array}\PY{p}{,} \PY{n}{xclara}\PY{p}{)}
        \PY{n}{xclara} \PY{o}{=} \PY{n}{xclara}\PY{o}{\PYZsq{}}
        \PY{c}{\PYZsh{}xclara\PYZus{}kmeans = kmeans(Float64.(xclara), 3)}
        \PY{n}{xclara\PYZus{}kmeans} \PY{o}{=} \PY{n}{kmeans}\PY{p}{(}\PY{n}{xclara}\PY{p}{,} \PY{l+m+mi}{3}\PY{p}{)}
\end{Verbatim}


    \paragraph{DBSCAN: otro algoritmo de clustering. Este no recibe la
cantidad de clusters como hiperparámetro y tiene la capacidad de manejar
mejor los puntos entre
clusters}\label{dbscan-otro-algoritmo-de-clustering.-este-no-recibe-la-cantidad-de-clusters-como-hiperparuxe1metro-y-tiene-la-capacidad-de-manejar-mejor-los-puntos-entre-clusters}

    \begin{Verbatim}[commandchars=\\\{\}]
{\color{incolor}In [{\color{incolor}27}]:} \PY{k}{using} \PY{n}{Distances}
         \PY{n}{dclara} \PY{o}{=} \PY{n}{pairwise}\PY{p}{(}\PY{n}{SqEuclidean}\PY{p}{(}\PY{p}{)}\PY{p}{,} \PY{n}{xclara}\PY{p}{)}\PY{p}{;}
         \PY{c}{\PYZsh{} parametros: matriz de distancias, el radio de un cluster, }
         \PY{c}{\PYZsh{} mínimo número de puntos que forman un cluster}
         
         \PY{n}{xclara\PYZus{}dbscan} \PY{o}{=} \PY{n}{dbscan}\PY{p}{(}\PY{n}{dclara}\PY{p}{,} \PY{l+m+mi}{10}\PY{p}{,} \PY{l+m+mi}{40}\PY{p}{)}\PY{p}{;}
         \PY{c}{\PYZsh{} devuelve cantidad de puntos para cada cluster encontrado}
         \PY{n}{xclara\PYZus{}dbscan}\PY{o}{.}\PY{n}{counts}
\end{Verbatim}


    \begin{Verbatim}[commandchars=\\\{\}]

        UndefVarError: xclara not defined

        

    \end{Verbatim}

    \paragraph{KNN}\label{knn}

    \section{Estadisticas}\label{estadisticas}

    Index TIOBE Mayo 2018

    Populridad en Githut en 2018

    Popularidad de lenguajes en ofertas laborales en cuanto a machine
learning o data science en 2016

    Pero cuando cambiamos a la popularidad relativa.

    \section{Comparacion otros lenguajes}\label{comparacion-otros-lenguajes}

    Julia vs Python

Ventajas de Juila:

\begin{itemize}
\item
  Rápido por default.
\item
  Sintaxis amigable para matemáticas.
\item
  No se pierde el manejo automatico de memoria.
\item
  Paralelismo.
\end{itemize}

    Ventajas de Python:

\begin{itemize}
\item
  Los arrays de Julia son indexados a partir del 1.
\item
  Julia todavía es un lenguaje muy nuevo.
\item
  Python tiene más paquetes creados para el lenguaje.
\item
  Python tiene una comunidad más grande.
\end{itemize}

    \paragraph{Comparación con Python: cálculo de la traza de una matriz de
10mil x
10mil}\label{comparaciuxf3n-con-python-cuxe1lculo-de-la-traza-de-una-matriz-de-10mil-x-10mil}

    \begin{Verbatim}[commandchars=\\\{\}]
{\color{incolor}In [{\color{incolor} }]:} \PY{c}{\PYZsh{} Ojo cuando corran esto. Puede que se rompa todo. Capaz lo mejor}
        \PY{c}{\PYZsh{} sea llevarlo ya resuelto en una imagen}
        \PY{k}{using} \PY{n}{Distributions}
        \PY{n}{m} \PY{o}{=} \PY{l+m+mi}{10000}
        \PY{n}{matrix} \PY{o}{=} \PY{n}{rand}\PY{p}{(}\PY{n}{Uniform}\PY{p}{(}\PY{l+m+mf}{0.0}\PY{p}{,} \PY{l+m+mf}{10.0}\PY{p}{)}\PY{p}{,} \PY{n}{m}\PY{p}{,} \PY{n}{m}\PY{p}{)}
\end{Verbatim}


    \begin{Verbatim}[commandchars=\\\{\}]
{\color{incolor}In [{\color{incolor} }]:} \PY{n}{n} \PY{o}{=} \PY{l+m+mi}{0}\PY{p}{;}
        \PY{k}{for} \PY{n}{i} \PY{k+kp}{in} \PY{p}{[}\PY{l+m+mi}{1}\PY{o}{:}\PY{l+m+mi}{100}\PY{p}{]}
            \PY{n}{tic}\PY{p}{(}\PY{p}{)}
            \PY{n}{trace}\PY{p}{(}\PY{n}{matrix}\PY{p}{)}
            \PY{n}{aux} \PY{o}{=} \PY{n}{toq}\PY{p}{(}\PY{p}{)}
            \PY{n}{n} \PY{o}{=} \PY{n}{n} \PY{o}{+} \PY{n}{aux}
        \PY{k}{end}
        
        \PY{n}{n}\PY{o}{/}\PY{l+m+mi}{100}
\end{Verbatim}


    Python:

    \section{Casos de estudios}\label{casos-de-estudios}

    En 2015, el banco de reserva federal de nueva york uso julia para hacer
modelos de la economía de los estados unidos.

Notaron que el modelo hecho con el lenguaje era 10 veces más rapido que
el anterior (hecho con MATLAB) dado por:

\begin{itemize}
\item
  Un sistema flexible y potente que proporciona una forma natural de
  estructurar y simplificar la base de código.
\item
  Multiple dispatch, lo que les permitió escribir un código más
  genérico.
\item
  Un potente compilador que aumentó el rendimiento.
\end{itemize}

    El proyecto Celste que cataloga estrellas y galaxias en el Apache Point
Observatory de Nevo Mexico logró un gran numero de hitos:

\begin{itemize}
\item
  Logró rendimiento pico de 1.54 petaFLOPS usando 1.3 millones de
  threads.
\item
  Agregó \textasciitilde{}178 terabytes de datos en imagenes.
\item
  Produjo parametros estimados para 188 millones de estrellas y galaxias
  en 14.6 minutos.
\item
  Proporcionó no solo estimaciones para las fuentes de luz sino que, por
  primera vez proporcionó una medida de principios de la calidad de la
  inferencia para cada fuente de luz.
\end{itemize}

    \section{Conclusion}\label{conclusion}


    % Add a bibliography block to the postdoc
    
    
    
    \end{document}
